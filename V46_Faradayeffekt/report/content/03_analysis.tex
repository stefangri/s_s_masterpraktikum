\section{Auswertung}

\subsection{Magnetfeldmessung}
Die aufgenommenen Daten zur Bestimmung des maximalen Magnetfeldes innerhalb des Elektromagneten 
sind in Tabelle~\ref{tab: messwerte_magnetfeld} aufgeführt. Eine grafische Darstellung der Abhängigkeit zwischen Verschiebung 
$z$ auf der Symmetrieachse der Anordnung und Magnetfeldstärke $B$ ist in Abbildung~\ref{fig: magnetfeld} einzusehen. 
Für die nachfolgenden Rechnungen wird der maximal gemessene Wert 
\begin{equation}
    B\ua{max} = \input{../analysis/results/result_max_B.tex}
\end{equation}
als vorherrschendes Magnetfeld in der Probe angenommen. Auf einen Fit der Daten wird verzichtet, 
da sehr feinschrittig gemessen wurde.

\begin{figure}
\centering
\includegraphics[scale = 1]{../analysis/results/magnetfeld.pdf}
\caption{Darstellung der Messwerte aus Tabelle~\ref{tab: messwerte_magnetfeld} zur Bestimmung des Zusammenhangs zwischen 
Magnetfeld $B$ und Position $z$. Die $z$-Achse ist dabei vollkommen willkürlich gewählt.}
\label{fig: magnetfeld}
\end{figure}

\begin{table}
	\centering
	\caption{Messwerte der Magnetfeldmessung zur Bestimmung des maximalen Wertes für $B$. Hierbei bezeichnet $z$ die Verschiebung entlang der Symmetrieachse des Elektromagneten.}
	\label{tab: messwerte_magnetfeld}
	\begin{tabular}{
		S[table-format=3.0]
		S[table-format=3.0]
		S[table-format=3.0]
		S[table-format=3.0]
		}
	\toprule
		{$z$\;/\;\si{\milli\meter}} &
		{$B$\;/\;\si{\milli\tesla}} &
		{$z$\;/\;\si{\milli\meter}} &
		{$B$\;/\;\si{\milli\tesla}} \\
	\midrule
		 85 &   4 &  109 &  409 \\
		 90 &  18 &  110 &  407 \\
		 91 &  26 &  111 &  403 \\
		 92 &  35 &  112 &  397 \\
		 93 &  48 &  113 &  389 \\
		 94 &  70 &  114 &  376 \\
		 95 &  101 &  115 &  357 \\
		 96 &  143 &  116 &  333 \\
		 97 &  196 &  117 &  297 \\
		 98 &  268 &  118 &  254 \\
		 99 &  320 &  119 &  203 \\
		 100 &  353 &  120 &  148 \\
		 101 &  373 &  121 &  105 \\
		 102 &  387 &  122 &  76 \\
		 103 &  395 &  123 &  54 \\
		 104 &  401 &  124 &  38 \\
		 105 &  406 &  125 &  28 \\
		 106 &  409 &  130 &   6 \\
		 107 &  410 &  135 &   1 \\
		 108 &  410 &  &  \\
	\bottomrule
	\end{tabular}
\end{table}




\subsection{Messung mit undotierter \ce{GaAs}-Probe}
Die Messungen mit der undotierten \ce{GaAs}-Probe als Referenzsystem für die dotierten Proben liefert die 
in Tabelle~\ref{tab: messwerte_ga_as_rein} eingetragenen Ergebnisse. Mit der Dicke der Probe $L = \SI{5.11}{\milli\meter}$ kann für jede 
Wellenlänge ein normierter Wert der Faradayrotation bestimmt werden. Die Messpunkte sind in Abbildung~\ref{fig: ga_as_rein}
dargestellt. Es zeigt sich ein abfallender Trend der Faradayrotation mit der Wellenlänge. Ein Datenpunkt weicht deutlich 
hiervon ab, was später weiter diskutiert wird.  

\begin{table} 
\centering 
\caption{Messwerte der reinen GaAs Probe. Die eingestellten Winkel am Gorniometer $\vartheta(\pm B)$ in Abhängigkeit der Wellenlänge $\lambda$, daraus berechnete Faradayrotation $\vartheta_F$ und auf die Länge der Probe normierte Faradayrotation $\vartheta_{\text{norm}}.$} 
\label{tab: messwerte_ga_as_rein} 
\begin{tabular}{S S S S S } 
\toprule  
{$\lambda / \si{ \micro\meter}$} & {$\vartheta(+B) / \si{ \degree}$} & {$\vartheta(-B) / \si{ \degree}$} & {$\vartheta_F / \si{ \degree}$} & {$\vartheta_{\text{norm}} / \si{ \radian \per \milli\meter}$} \\ 
\midrule  
1.06 & 66.50 & 90.50 & 24.00 & 0.082\\ 
1.29 & 69.58 & 86.33 & 16.75 & 0.057\\ 
1.45 & 72.67 & 86.17 & 13.50 & 0.046\\ 
1.72 & 73.42 & 83.08 & 9.67 & 0.033\\ 
1.96 & 69.75 & 77.08 & 7.33 & 0.025\\ 
2.16 & 69.42 & 75.25 & 5.83 & 0.020\\ 
2.34 & 46.00 & 51.00 & 5.00 & 0.017\\ 
2.51 & 17.67 & 33.67 & 16.00 & 0.055\\ 
2.65 & 62.00 & 68.17 & 6.17 & 0.021\\ 
\bottomrule 
\end{tabular} 
\end{table}

\begin{figure}
\centering
\includegraphics[scale = 1]{../analysis/results/fit_undot.pdf}
\caption{Grafische Darstellung der Messwerte aus Tabelle~\ref{tab: messwerte_ga_as_rein} zur Analyse der 
Faradayrotation der undotierten Probe. Auf Länge der Probe normierte 
Faradayrotation $\vartheta_{\text{Norm}}$ in Abhängigkeit des Wellenlängenquadrats $\lambda^2$. }
\label{fig: ga_as_rein}
\end{figure}


\subsection{Messung mit dotierten \ce{GaAs}-Proben}

Die Messwerte der dicken ($L = \SI{1.36}{\milli\meter}$, $N = \SI{1.2e18}{1/cm^3}$) und dünnen 
($L = \SI{1.296}{\milli\meter}$, $N = \SI{2.8e18}{1/cm^3}$) $n$-dotierten Probe sind in den 
Tabellen~\ref{tab: ga_as_dot_dick} und~\ref{tab: ga_as_dot_duenn} aufgeführt. Die Faradayrotation $\vartheta_F$ wird jeweils normiert und von diesem 
Wert für alle Wellenlängen die normierte Faradayrotaion der undotierten Probe abgezogen. So kann 
die Faradayrotation $\Delta \vartheta\ua{norm}$, die durch die Leitungselektronen bedingt ist, untersucht werden. Die berechneten Werte 
sind in Abbildung~\ref{fig: ga_as_dot} grafisch dargestellt. Es zeigt sich in beiden Fällen eine lineare Abhängigkeit zwischen $\lambda^2$
und $\Delta \vartheta$, wie aus Gleichung~\eqref{} hervorgeht. Im Fall der dicken Probe weicht ein Datenpunkt deutlich 
von dem linearen Trend ab. Dieser wird für die lineare Ausgleichsrechnung nicht berücksichtigt. Ein Fit der Daten an 
ein lineares Modell der Gestalt 
\begin{equation}
    \Delta\vartheta\ua{norm} = A \lambda^2 + B
\end{equation}
liefert für die dicke Probe 
\begin{equation}
    A\ua{dick} = \input{../analysis/results/result_slope_dick.tex}, 
    \quad B\ua{dick} = \input{../analysis/results/result_offset_dick.tex}
\end{equation}
und für die dünne Probe 
\begin{equation}
        A\ua{dick} = \input{../analysis/results/result_slope_duenn.tex}, 
    \quad B\ua{dick} = \input{../analysis/results/result_offset_duenn.tex}
\end{equation}
Aus der Steigung $A$ lässt sich gemäß Formel~\eqref{} die Effektive Masse bestimmen. Für die dicke Probe ergibt sich 
\begin{align}
\begin{aligned}
m^*\ua{dick} &= \input{../analysis/results/result_mass_dick.tex} \\
\frac{m^*\ua{dick}}{m\ua{e}} &= \input{../analysis/results/result_mass_ratio_dick.tex}.
\end{aligned}
\end{align}
Und analog für die dünne Probe 
\begin{align}
\begin{aligned}
m^*\ua{dünn} &= \input{../analysis/results/result_mass_duenn.tex} \\
\frac{m^*\ua{dünn}}{m\ua{e}} &= \input{../analysis/results/result_mass_ratio_duenn.tex}.
\end{aligned}
\end{align}


\begin{table} 
\centering 
\caption{Messwerte der dotierten GaAs Probe mit $N = \SI{2.8e18}{\per\centi\meter^3}$ und $L = \SI{1.296}{\milli\meter}$. Die eingestellten Winkel am Gorniometer $\vartheta(\pm B)$ in Abhängigkeit der Wellenlänge $\lambda$, daraus berechnete Faradayrotation $\vartheta_F$ und auf die Länge der Probe normierte Faradayrotation $\Delta \vartheta_{\text{norm}}$ (abzüglich der Faradayrotation der reinen GaAs Probe).} 
\label{tab: messwerte_ga_as_dot_duenn} 
\begin{tabular}{S S S S S } 
\toprule  
{$\lambda / \si{ \micro\meter}$} & {$\vartheta(+B) / \si{ \degree}$} & {$\vartheta(-B) / \si{ \degree}$} & {$\vartheta_F / \si{ \degree}$} & {$\Delta \vartheta_{\text{norm}} / \si{ \radian \per \milli\meter}$} \\ 
\midrule  
1.06 & 72.42 & 83.92 & 5.75 & 0.036\\ 
1.29 & 72.58 & 81.25 & 4.33 & 0.030\\ 
1.45 & 73.75 & 83.00 & 4.62 & 0.039\\ 
1.72 & 72.50 & 82.00 & 4.75 & 0.047\\ 
1.96 & 66.67 & 78.00 & 5.67 & 0.064\\ 
2.16 & 65.58 & 77.67 & 6.04 & 0.071\\ 
2.34 & 41.17 & 54.92 & 6.88 & 0.084\\ 
2.51 & 24.17 & 39.42 & 7.62 & 0.075\\ 
2.65 & 59.92 & 74.00 & 7.04 & 0.084\\ 
\bottomrule 
\end{tabular} 
\end{table}

\begin{table} 
\centering 
\caption{Messwerte der dotierten GaAs Probe mit $N = \SI{1.2e18}{\per\centi\meter^3}$ und $L = \SI{1.36}{\milli\meter}$. Die eingestellten Winkel am Gorniometer $\vartheta(\pm B)$ in Abhängigkeit der Wellenlänge $\lambda$, daraus berechnete Faradayrotation $\vartheta_F$ und auf die Länge der Probe normierte Faradayrotation $\Delta \vartheta_{\text{norm}}$ (abzüglich der Faradayrotation der reinen GaAs Probe).} 
\label{tab: messwerte_ga_as_dot_dick} 
\begin{tabular}{S S S S S } 
\toprule  
{$\lambda / \si{ \micro\meter}$} & {$\vartheta(+B) / \si{ \degree}$} & {$\vartheta(-B) / \si{ \degree}$} & {$\vartheta_F / \si{ \degree}$} & {$\Delta \vartheta_{\text{norm}} / \si{ \radian \per \milli\meter}$} \\ 
\midrule  
1.06 & 73.83 & 83.50 & 9.67 & 0.042\\ 
1.29 & 74.50 & 81.50 & 7.00 & 0.033\\ 
1.45 & 75.83 & 81.75 & 5.92 & 0.030\\ 
1.72 & 75.42 & 81.50 & 6.08 & 0.045\\ 
1.96 & 70.50 & 76.42 & 5.92 & 0.051\\ 
2.16 & 68.83 & 75.75 & 6.92 & 0.069\\ 
2.34 & 44.67 & 51.92 & 7.25 & 0.076\\ 
2.51 & 26.08 & 32.67 & 6.58 & 0.030\\ 
2.65 & 62.83 & 70.83 & 8.00 & 0.082\\ 
\bottomrule 
\end{tabular} 
\end{table}

\begin{figure}
\centering
\includegraphics[scale = 1]{../analysis/results/fit_dot.pdf}
\caption{Grafische Darstellung der Messwerte aus den Tabellen~\ref{tab: messwerte_ga_as_dot_duenn} und~\ref{tab: messwerte_ga_as_dot_dick} 
zur Untersuchung der 
Wellenlängenabhängigkeit der Faradayrotaion der dünnen und dicken dotierten \ce{GaAs}-Probe. Auf Länge der Probe normierte 
Faradayrotation $\Delta \vartheta_{\text{Norm}}$ in Abhängigkeit des Wellenlängenquadrats $\lambda^2$. Für die dicke Probe 
wurde der ignorierte Datenpunkt farblich anders markiert.}
\label{fig: ga_as_dot}
\end{figure}