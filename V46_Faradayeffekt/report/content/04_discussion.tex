\section{Diskussion}
Der lineare Zusammenhang zwischen der Faradayrotation und dem Quadrat der Wellenlänge des 
Lichtes konnte im Experiment gezeigt werden (siehe Abbildung~\ref{fig: ga_as_dot}). Das Konzept der effektiven 
Masse und das Semiklassische Modell für freie Ladungsträger in Halbleitern wird dadurch experimentell 
bestätigt.

Die beiden gewonnen Werte für die effektive Masse der freien Ladungsträger in $n$-dotiertem 
\ce{GaAs} ($m^*\ua{dick} = \input{../analysis/results/result_mass_ratio_dick.tex}m\ua{e}$
und $m^*\ua{duenn} = \input{../analysis/results/result_mass_ratio_duenn.tex}m\ua{e}$)
liegen in einem ähnlichen Bereich. 
In der Literatur findet man den Wert $\num{0.067}m\ua{e}$ \cite{semiconductors}. Die experimentellen 
Ergebnisse stimmen 
also sehr gut überein und der Aufbau ist als geeignetes Instrument zur Bestimmung der effektiven Masse einzustufen.
Hierbei ist jedoch zu erwähnen, dass die Winkelmessung hin zu größeren Wellenlängen erschwert ist, da die 
Signalstärke abnimmt und so das Minimum am Oszilloskop nur schwer erkannt werden kann. Hierdurch kann auch 
der offenbar fehlerhafte Datenpunkt in der Abbildung~\ref{fig: ga_as_dot} erklärt werden. 