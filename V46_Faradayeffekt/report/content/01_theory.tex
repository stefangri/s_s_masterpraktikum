\setcounter{page}{1}
\section*{Zielsetzung}
Der Versuch $V46$ verwendet den \emph{Faraday-Effekt}, um die \emph{effektive Masse} von
Galliumarsenid zu bestimmen. Hierzu werden eine undotierte und zwei n-dotierte
Formen des Halbleiters verwendet.

\section{Theorie}
Zu Beginn wird die Verwendung einer effektiven Masse motiviert.
Bevor der Faraday-Effekt beschrieben wird, wird das Phänomen der zirkularen
Doppelbrechung erläutert.

\subsection{Effektive Masse}
 Bänder eines kristalls sind meist in einer komplexen Bandstruktur angeordnet.
Hierdurch ist eine exakte mathemaische Beschreibung meist schwierig und bedarf
Approximationen. Vorteilhafterweise beschreiben die verwendeten Näherung oftmals
die auftretenden physikalischen Phänomene sehr gut. Bei Halbleitern ist eine
dieser Approximationen, die Betrachtung des Leitungsbandes um das Minium herum.
In der Umgebung um das Minimum kann die Energie des Bandes $\epsilon$
in Abhängigkeit mit dem Wellenzahlvektor $\vec{k}$ genährt werden als:
\begin{equation}
  \label{eq:gleichung_energie}
  \epsilon(\vec{k})=\epsilon(0) + \frac{1}{2}\sum_{i=1}^3 \left.\frac{\partial^2 \epsilon}{\partial k_i^2}\right|_{k=0}k_i^2 + \mathcal{O}(k^3).
\end{equation}
Dieser Term eröffnet die Einführung der effektiven Masse
\begin{equation}
  \label{eq:effetive_masse}
  m*_i := \frac{\hbar^2}{\left.\frac{\partial^2 \epsilon}{\partial k_i^2}\right|_{k=0}}.
\end{equation}
Der Vorteil in der Verwendung der effektiven Massse liegt in der Tatsache das
die Definition \eqref{eq:effetive_masse} die Periodiztät des  Kristallpotentials
$V(\vec{r})$ mitberücksichtigt. Somit kann der Hamilton Operator für ein Kristallelektron,
mit Hilfe der effektiven Masse, in einen Hamilton Operator für ein freies
Teilchen überführt werden:
\begin{equation*}
  \hat{H}:\quad \frac{\hbar^2}{2\map{m_e}}\nabla^2 + V(\vec{r}) \quad \rightarrow \quad   \frac{\hbar^2}{2m^*}\nabla^2
\end{equation*}
Ein weiterer Vorteil in der Beschreibung Dynamik mit der effektiven Massse ist,
dass bei anlegen eines externen elektrischen und magnetischen Feldes,
vorrausgesetzt es ist klein, das eine klassiche Newtonsche Betrachtung möglich ist.
Diese Beschaffenheit wird insbesondere im übernächsten Kapitel von nutzen sein.
\subsection{Zirkulare Doppelbrechung}
Wird die Polarisationsebene von linear polarisiertes Licht $E(z)$ bei der Propegation
durch einen Kristall (der Länge $L$) gedreht, wird von zirkularer Doppelbrechung gesprochen.
Phänomenologisch erklären lässt sich der Effekt unter der Annahme das sich im
Kristall die Phasengeschwindigkeit für links- und rechtszirkular polarisiertes
Licht $E\ua{L}, E\ua{R}$ unterscheiden. Diese Eigenschaft wirkt sich auf die Polarisationsebene
von linear polarisierten Licht aus, weil solches durch eine Linearkombination
von links- und rechtszirkularen Anteilen dargestellt werden kann:
\begin{equation}
  \label{eq:superpos_linear}
\vec{E}(z) = \frac{1}{2}( \vec{E}\ua{R}(z) + \vec{E}\ua{L}(z)), \quad k\ua{R}\neq k\ua{L}
\end{equation}
Wobei die links- und rechtszirkularen definiert sind als
\begin{align}
  \label{eq:def_rechts_links}
  \begin{aligned}
  \vec{E}\ua{R}(z) &= \left(E_0 \vec{x}_0 - \map{i}E_0\vec{y}_0\right)\exp\left(\map{i}{k}\ua{R}z\right)\\
  \vec{E}\ua{L}(z) &= \left(E_0 \vec{x}_0 + \map{i}E_0\vec{y}_0\right)\exp\left(\map{i}{k}\ua{L}z\right)
\end{aligned}
\end{align}
Die Defintionen \eqref{eq:def_rechts_links} werden in die Gleichung \eqref{eq:superpos_linear}
eingesetzt. Zusätzlich werden die Winkel
\begin{align}
  \label{eq:winkel}
  \begin{aligned}
    \Psi&:=\frac{L}{2}(k\ua{R}+k\ua{L}) \\
    \vartheta &:= \frac{L}{2}(k\ua{R}-k\ua{L}) \overset{k_i=\frac{n_i\omega}{\map{c_0}}}{=} \frac{L\omega}{2\map{c_0}}\left(n\ua{R}-n\ua{L}\right)
\end{aligned}
\end{align}
eingeführt, wobei $n$ der jeweilige Brechungsindex ist.
Nach ein wenig Rechnung (nachlesbar in Quelle \cite{anleitungv46}) ergibt sich ein
Ausdruck für das aus dem Kristall austredene Licht:
\begin{equation*}
  \vec{E}(L)=E_0 \exp(\map{i}\Psi)\left(\cos(\vartheta) \vec{x}_0 + \sin(\vartheta)\vec{y}_0\right)
\end{equation*}
