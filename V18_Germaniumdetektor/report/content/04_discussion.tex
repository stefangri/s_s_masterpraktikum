\section{Diskussion}
In diesem Kapitel sollen die Messergebnisse diskutiert werden.

Trotz linearer Zusammenhang zwischen Energie und Channel zeigt der Fitparameter
$b$ einen auffällig großen Fehler. Eventuell wird der Fehler durch
die Fitfunktion \textsc{curve\_fit} verursacht und sollte mit einer anderen
Regressionsfunktion geprüft werden. Der große Fehlbereich hat insbesondere auf
spätere Umrechnung in kleine Energien signifikante Auswirkungen bezüglich der Genauigkeit.
Aus diesem Grund kann die Qualität aller nachfolgenden Regressionen an den
untransformierten Größen festgestellt werden.

Die bestimmte Energie des Vollenergiepeaks von $^{137}\ce{Cs}$ stimmt mit dem
Literaturwert $\mu\ua{vp, theo}=\SI{661}{\kilo\eV}$ \cite{chartofnuclieds}
überein und veranschaulicht deutlich die Präzision des verwendeten Detektors.
Jedoch ist das Auflösungsvermögen des Germaniumdetektors nicht maximal, was sich
insbesondere bei Untersuchung der Halbwertsbreiten zeigt. Verursacht wird
die Halbwertsbreite eventuelle durch thermisches Rauschen im Detetektor oder
Verstärker oder durch rückstreuende Teilchen die eine Energie besitzen bei der
der Photoeffekt wieder wharscheinlicher wird. Weiterhin ist der signifikante Unterschied
zwischen $E\ua{rück}$ und $E\ua{rück, calc}$ aufällig. Vermutlich wird der
Unterschied durch einen nicht genaues ablesen des Rückstreupeaks verursacht.
Das Ablesen könnte durch eine Regression verbessert werden.

Die beiden unbekannten Quellen konnte mit einer hohen Genauigkeit bestimmt werden.
Dies spricht abermals für die Präzision eines Germaniumdetektors.
