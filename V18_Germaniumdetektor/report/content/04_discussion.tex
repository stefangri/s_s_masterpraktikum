\section{Diskussion}
In diesem Kapitel sollen die Messergebnisse diskutiert werden.

Trotz linearer Zusammenhang zwischen Energie und Channel zeigt der Fitparameter
$b$ einen auffällig große Unsicherheit. Eventuell wird die Unsicherheit durch
die Fitfunktion \textsc{curve\_fit} verursacht. Insbesondere zeigt sich bei der
Betrachtung der Kovarianzmatrix $\tens\sigma$ keine signifikante Korrelation
der beiden Fitparameter:
\begin{equation*}
  \tens{\sigma} = \begin{pmatrix} \num{2.9e-6} & \num{-5.1e-3} \\ \num{-5.1e-3} & \num{11.8} \end{pmatrix}
\end{equation*}
Die Unsicherheit sollte mit einer anderen Regressionsfunktion geprüft werden.
Weiterhin stellen die bestimmten Effizienzen $Q$ lediglich eine Näherung da,
weil der in die Berechnung einfließende Parameter $\frac{\Omega}{4\pi}$ mittels
einer Approximation ermittelt wird.

Die bestimmte Energie des Vollenergiepeaks von $^{137}\ce{Cs}$ stimmt mit dem
Literaturwert annähernd $\mu\ua{vp, theo}=\SI{661}{\kilo\eV}$ \cite{chartofnuclieds}
überein (Abweichung: $0.15\%$) und veranschaulicht deutlich die Präzision des verwendeten Detektors.
Beim Vergleich der gemmessen Peakenergien der anderen Isotope zeigt sich, dass
die Präzision insbesondere bei höheren Energien konstant ist. Lediglich bei kleineren
Energien, wie etwa bei $^{133}\ce{Ba}$ bei der Energie $E_\gamma=\SI{83.58}{\kilo\eV}$, zeigt sich eine
Abweichung zum Literaturwert von $\approx 3\%$.
Weiterhin ist der Unterschied
zwischen $E\ua{rück}$ und $E\ua{rück, calc}$ aufällig. Vermutlich wird der
Differenz durch ein nicht genaues ablesen des Rückstreupeaks verursacht.
Das Ablesen könnte durch eine Regression verbessert werden.
Außerdem unterscheidt sich das Flächenverhältnis $\frac{I\ua{compton}}{I\ua{peak}}$
signifikant von dem Verhältnis $\frac{P\ua{compton}}{P\ua{Photo}}$, obwohl beide
ein Wahrscheinlichkeitsmaß für die Wechselwirkungsprozesse darstellen.
Erklärt werden kann die Diskrepanz dadurch, dass $\frac{P\ua{compton}}{P\ua{Photo}}$
lediglich $1$-fache Photoeffekte berücksichtigt. In der Realität
tritt der Photoeffekt, trotz hoher Energien, jedoch mehrfach auf. Diese Tatsache lässt sich anhand
Abbildung \ref{fig:extinktionskoeffizient} erläutern. Wechselwirkt ein hochenergetisches
Photon im Material verliert dieses Energie. Der Energieverlust führt wiederum,
zu einer höheren Wahrscheinlichkeit, dass das Photon über den Photoeffekt wechselwirkt.
Somit ist es zum Beispiel auch möglich, dass ein Photon mehrfach über den Photoeffekt
wechselwirkt. Aus diesem Grund ist die Fläche $I\ua{peak}$ deutlich vergrößert.

Die beiden unbekannten Quellen konnte mit einer hohen Genauigkeit bestimmt werden.
Dies spricht abermals für die Präzision eines Germaniumdetektors.
