\section{Auswertung}
Für die folgenden Analysen wird aus dem \textsc{python} Paktet \textsc{scipy.optimize}~\cite{scipy}
die Funktion \textsc{curve\_{fit}} für Regressionsberechnung verwendet.

\subsection{Energiekalibrierung des Germaniumdetektors}\label{sec:kalibrierung}
\FloatBarrier
Die Kalibrierung wird mit Hilfe einer Europium $^{152}\ce{Eu}$ Probe
durchgeführt. Das nach einer Messzeit von $t\ua{measurment}=\SI{3380}{\second}$
aufgenommene Energiespektrum ist in Abbildung \ref{fig:spectrum_europ} dargestellt.
Der Breich größer $4000$ wurde in der Abbildung abgeschnitten, da sich in diesem
Bereich kein zusätzlicher Peak befindet.
\begin{figure}
  \centering
  \includegraphics[width=0.8\textwidth]{../analysis/plots/europium/spektrum_index.pdf}
  \caption{Gemessene Spektrum von Europium $^{152}\ce{Eu}$.}
  \label{fig:spectrum_europ}
\end{figure}
In der Abbildung \ref{fig:spectrum_europ} sind zusätzlich die mit
\textsc{scipy.signal.find\_{peaks}} ermittelten Peaks eingezeichnet.
An jeden Peak wird eine Gausfunktion
\begin{equation}
  \label{eq:gaus}
  G(x)=A\exp{\left(-\frac{1}{2}\left(\frac{x-\mu}{\sigma}\right)^2\right)}
\end{equation}
angepasst, um den Hochpunkt des Peaks gegeben durch den Parameter $\mu$ zu
bestimmen. Die Parameter befinden sich in Tabelle \ref{tab: results_peaks_eu}.
\begin{table}
\centering
\caption{Bestimmte Eigenschaften der Peaks von $^{152}\ce{Eu}$.}
\label{tab: results_peaks_eu}
\begin{tabular}{S S[table-format=2.2]@{${}\pm{}$} S[table-format=2.2] S[table-format=1.2]@{${}\pm{}$} S[table-format=1.2] S[table-format=5.2]@{${}\pm{}$} S[table-format=1.2] }
\toprule
{$Channelnummer $} & \multicolumn{2}{c}{$A \:/\:\newline \map{Counts}$} & \multicolumn{2}{c}{$\sigma \:/\:\newline \map{Channelnummer}$} & \multicolumn{2}{c}{$\mu \:/\:\newline \map{Channelnummer}$} \\
\midrule
309 & 3975.18 & 67.24 & 1.20 & 0.02 & 308.85 & 0.02\\
614 & 594.60 & 23.55 & 1.50 & 0.07 & 613.93 & 0.07\\
861 & 1273.13 & 14.78 & 1.53 & 0.02 & 860.98 & 0.02\\
1027 & 85.16 & 7.82 & 2.22 & 0.24 & 1026.96 & 0.24\\
1108 & 98.87 & 6.89 & 2.22 & 0.18 & 1108.28 & 0.18\\
1939 & 181.41 & 6.69 & 2.49 & 0.11 & 1939.53 & 0.11\\
2158 & 51.85 & 4.58 & 3.78 & 0.39 & 2158.78 & 0.39\\
2398 & 121.34 & 3.64 & 3.29 & 0.11 & 2398.88 & 0.11\\
2701 & 67.28 & 4.11 & 3.93 & 0.28 & 2701.51 & 0.28\\
2767 & 81.30 & 3.29 & 3.74 & 0.17 & 2766.37 & 0.17\\
3501 & 86.81 & 3.89 & 3.91 & 0.20 & 3501.45 & 0.20\\
\bottomrule
\end{tabular}
\end{table}

Die zu den Peaks gehörige Kanalnummer wird mit der, aus der Anleitung \cite{anleitungv18}
bekannten, charakteristischen Energie zugeordnet.
Damit die gesamte $x$-Achse auf Einheiten der Energie transformiert werden kann,
wird eine lineare Regression an die Funktion
\begin{equation*}
  g(x)=mx+b
\end{equation*}
durchgeführt.
Aus der Regressionsrechnung ergeben sich die folgenden Parameter
\begin{equation}
  \label{eq:Kanal_energie_trafo}
  m=\num{0.4018\pm0.0017}\,\frac{\si{\kilo\eV}}{\map{Kanalnummer}}, \quad b=\SI{-2.2\pm3.4}{\kilo\eV}.
\end{equation}
Verwendet wurde für die Regressions die in \cite{anleitungv18} angebenen Energien,
der jeweiligen Peaks. Die dazugehörigen Werte befinden sich in Tabelle 2 %Irgendwie hat die Ref nicht funktioniert.
unter $E\ua{\gamma,theo}$.
Auffällig ist die Unsicherheit des Parameters $b$.
In der Abbildung \ref{fig:skalen_trafo_fit} ist der Fit mit den Fitpunkten dargestellt.
\begin{figure}
  \centering
  \includegraphics[width=0.8\textwidth]{../analysis/plots/europium/skalen_trafo_fit.pdf}
  \caption{Fitpunkte und Regressionsgerade der Energietransformation.}
  \label{fig:skalen_trafo_fit}
\end{figure}
Die aus der Transformation resultierenden Energien sind in der Tabelle 2 %Irgendwie hat die Ref nicht funktioniert.
aufgelistet.
\begin{table}
\centering
\caption{Energiewerte der Peaks von $^{152}\ce{Eu}$.}
\label{tab: energy__peaks_eu}
\begin{tabular}{S S S[table-format=3.2]@{${}\pm{}$} S[table-format=1.2] }
\toprule
{$Channel$} & {$E_{\gamma,theo}\: /\: \si{ \kilo\eV }$} & \multicolumn{2}{c}{$E_{\gamma} \:/\: \si{ \kilo\eV }$} \\
\midrule
309 & 121.78 & 123.81 & 2.88\\
614 & 244.70 & 246.10 & 2.98\\
861 & 344.30 & 345.13 & 3.11\\
1027 & 411.96 & 411.66 & 3.22\\
1108 & 443.96 & 444.26 & 3.27\\
1939 & 778.90 & 777.47 & 4.02\\
2158 & 867.37 & 865.36 & 4.26\\
2398 & 946.08 & 961.61 & 4.53\\
2701 & 1085.90 & 1082.92 & 4.88\\
2767 & 1112.10 & 1108.92 & 4.96\\
3501 & 1408.00 & 1403.58 & 5.88\\
\bottomrule
\end{tabular}
\end{table}

Die in diesem Abschnitt bestimmten Parameter werden in späteren Teilen der Auswertung
ohne weitere Anmerkung verwendet.
\FloatBarrier

\FloatBarrier
Der Detektor kann nicht jede Energie mit derselben Vollenergienachweiseffizienz
nachweisen, deshalb wird diese im Folgenden untersucht. Hierzu wird die Formel
\eqref{eq: zählrate} nach $Q$ umgestellt. Die Größe $Z$ wird durch Summation der Kanalinhalte
um ein Peak bestimmt. Dabei ist zu beachten, dass $Z$ durch die Messzeit $t\ua{measurment}$
dividiert wird. Die Übergangswahrscheinlichkeiten werden der Anleitung \cite{anleitungv18}
entnommen. Der Raumwinkel wird mit Gleichung \eqref{eq: omega} berechnet. Hierbei wurde
für den Abstand von Probe und Detektor der Wert $a=\SI{730}{\milli\meter}+\SI{150}{\milli\meter}=\SI{881}{\milli\meter}$ und für
den Radius der Querschnittfläche des Detektors $r=\SI{27.5}{\milli\meter}$
verwendet. Der Abstand ergibt sich hierbei aus der Distanz der Probe zur
Schutzhaube des Detekotrs (1. Summand) und aus der Dicke der Haube (2. Summand).
Damit folgt für den Raumwinkel
\begin{equation}
  \label{eq:winkelverteilung}
  \frac{\Omega}{4\pi}=\num{0.0155}.
\end{equation}
Der Raumwinkel wird ohne Unischerheit berücksichtigt, weil es sich bei der Formel
\eqref{eq: omega} um eine Näherung handelt.
Die anfängliche Aktivität der Probe lag nach Angaben der Anleitung \cite{anleitungv18}
am 01.10.2000 bei $A_0=\SI{4130\pm60}{\becquerel}$. Mit dem Zerfallsgesetz kann die
Aktivität der Quelle am 29.10.2018 bestimmt werden:
\begin{equation}
  A(t\ua{29.10.2018})=A_0\exp\left(-\frac{t\ua{29.10.2018}\log(2)}{\tau\ua{1/2}}\right) = \SI{1636\pm24}{\becquerel}.
\end{equation}
Bei der Berechnung wurde die in der Anleitung \cite{anleitungv18} angegebene
Halbwertszeit von $^{152}\ce{Eu}$, $\tau\ua{1/2}=\SI{4943 \pm 5 }{\day}$ genutzt.
Eingetragen sind die berechneten Effizienzen der Probe in Tabelle \ref{tab: results_europium}.
\begin{table}
\centering
\caption{Bestimmten Energie und Effizienzwerte.}
\label{tab: results_europium}
\begin{tabular}{S S S[table-format=1.4]@{${}\pm{}$} S[table-format=1.4] }
\toprule
{$Channel$} & {$Energie \: / \: \si{ \kilo\eV}$} & \multicolumn{2}{c}{$Q$} \\
\midrule
309 & 121.78 & 0.4910 & 0.0071\\
614 & 244.70 & 0.3413 & 0.0050\\
861 & 344.30 & 0.2127 & 0.0031\\
1027 & 411.96 & 0.2273 & 0.0033\\
1108 & 443.96 & 0.1857 & 0.0027\\
1939 & 778.90 & 0.0882 & 0.0013\\
2158 & 867.37 & 0.0900 & 0.0013\\
2398 & 946.08 & 0.0592 & 0.0009\\
2701 & 1085.90 & 0.0512 & 0.0007\\
2767 & 1112.10 & 0.0476 & 0.0007\\
3501 & 1408.00 & 0.0320 & 0.0005\\
\bottomrule
\end{tabular}
\end{table}

Die nachfolgenden Regressionsrechnungen werden an Energien $E_\gamma>\SI{150}{\kilo\eV}$
durchgeführt.
An die bestimmten Vollenergienachweiseffizienzen wird eine Funktion der Form
\begin{equation*}
  Q_1(E)=A_1\cdot E^{B_1}
\end{equation*}
angepasst. Die Parameter folgen aus der Regressionsrechnung als:
\begin{equation}
\label{eq:effizienz_Q_1}
A_1 = \SI{2.1\pm1.1 e2}{\per\kilo\eV}, \quad B_1=\num{-1.17\pm 0.09}.
\end{equation}
.Außerdem wird an die Datenpunkte der Tabelle \ref{tab: results_europium}
eine Funktion der Form
\begin{equation*}
  Q_2(E)=A_2\exp\left(-B_2 E\right)+C
\end{equation*}
angepasst. Motviert werden kann die Verwendung einer zweiten Regressionsfunktion, durch
die signifikante Unsicherheit der Parameter $A_1$ und $B_1$.

Aus der Regressionsrechnung ergeben sich die folgenden Parameter
\begin{equation}
\label{eq:effizienz_Q_2}
A_2 = \num{0.64\pm0.09}, \quad B_2=\SI{32\pm6 e-4}{\per\kilo\eV}, \quad C = \num{0.030\pm0.016}.
\end{equation}
Dargestellt sind beide Fitfunktion und die Datenpunkte in Abbildung \ref{fig:effizienz}.
\begin{figure}
  \centering
  \includegraphics[width=0.7\textwidth]{../analysis/plots/europium/effizienz.pdf}
  \caption{Ermittelte Vollenergienachweiseffizienz $Q$ des Detektors.}
  \label{fig:effizienz}
\end{figure}
Für die späteren Aktivitätsbestimmungen wird die Effizienzfunktion $Q_2$ mit den
ermittelten Parametern ohne weitere Kommentare verwendet, weil diese einen kleinere
Unsicherheit aufweisen.
\FloatBarrier
\subsection{Untersuchung einer $^{137}{Cs}$ Quelle}
\FloatBarrier
Das nach einer Messzeit von $t\ua{measurement}=\SI{3694}{\second}$ gemessene
Energiespektrum ist in Abbildung \ref{fig:caesium_spektrum} dargestellt.
\begin{figure}
  \centering
  \includegraphics[width=0.7\textwidth]{../analysis/plots/caesium/caesium_spektrum.pdf}
  \caption{Gemessene Energiespektrum einer $^{137}{Cs}$ Quelle.}
  \label{fig:caesium_spektrum}
\end{figure}
Deutlich im Spektrum zu erkennen ist der Vollenergiepeak und die Comptokante.
Zunächst wird die Energie des Vollenergiepeak bestimmt. Hierzu wird an den Peak
eine Gausfunktion \eqref{eq:gaus} angepasst. Aus der Regressionsrechnung ergeben
sich die folgenden Parameter:
\begin{align*}
  a\ua{vp}&=\num{2407\pm29}\,\map{Counts}\\
  \mu\ua{vp}&=\SI{660\pm0.012}{\kilo\eV} \\
  \sigma\ua{vp}&=\SI{0.874\pm0.012}{\kilo\eV}
\end{align*}
Der Vollenergiepeak und der bestimmte Fit sind in Abbildung \ref{fig:Vollenergiepeak} eingezeichnet.
\begin{figure}
  \centering
  \includegraphics[width=0.7\textwidth]{../analysis/plots/caesium/photopeak.pdf}
  \caption{Vollenergiepeak der $^{137}{Cs}$ Quelle.}
  \label{fig:Vollenergiepeak}
\end{figure}
Zusätzlich sind in der Abbildung \ref{fig:Vollenergiepeak} die abgelesenen
Halbwerts- und Zehntelbreiten eingezeichnet:
\begin{align}
  \label{eq:halbwerts_zehntel_breite_gemessen}
  x\ua{1/2}&= \SI{2.27\pm 8.89}{\kilo\eV}\\
  x\ua{1/10}&=\SI{4.17\pm8.87}{\kilo\eV} \\
  \frac{x\ua{1/10}}{x\ua{1/2}} &=\num{1.85\pm 11.21}.
\end{align}
Hier bewirkt die Energietransformation eine signifikante Vergrößerung des Unsicherheitsbereichs.
Mit Hilfe des Parameters $\sigma$ des Vollenergiepeaks ist es möglich die
Breiten in Gleichung \eqref{eq:halbwerts_zehntel_breite_gemessen} gegenzuprüfen.
Hierfür wird die aus Quelle \cite{halbwertsbreite} und \cite{anleitungv18}
entnommene Formel verwendet:
\begin{equation}
  \label{eq:formel_umrechnung}
  x\ua{1/2,\sigma}=2\log(2)\sigma, \quad x\ua{1/10,\sigma}=1.823 x\ua{1/2}=3.646\log(2)\sigma
\end{equation}
Mit der Gleichung \eqref{eq:formel_umrechnung} folgt somit:
\begin{align}
  \label{eq:halbwerts_zehntel_breite_berechnet}
  x\ua{1/2}&= \SI{2.06\pm0.03}{\kilo\eV}\\
  x\ua{1/10}& = \SI{3.75\pm 0.05}{\kilo\eV}
  \frac{x\ua{1/10}}{x\ua{1/2}} &=\num{1.82\pm 0.01}
\end{align}
Die Gleichung \eqref{eq:formel_umrechnung} gilt jedoch nur unter der Annahme, dass
die Peaks Gausverteilt sind. Eine Annahme die sich mit den bestimmten Verhältnissen
$ \frac{x\ua{1/10}}{x\ua{1/2}}$ berechtigt ist.
Zusätzlich kann die Halbwertsbreite mit Gleichung
\begin{equation}
  \label{eq: energie_halbwertsbreiten}
  x\ua{1/2,f}=\Delta E_{1/2}=2.35\sqrt{0.1 E\ua{\gamma} E\ua{EL}}
\end{equation}
über die Vollenergiepeakenergie $E\ua{\gamma}$ berechnet werden.
Die Energie $E\ua{EL}=\SI{2.9}{\eV}$ bezeichnet die Anregungsenergie
eines Elektrons in Germanium.
Hierzu wird der Fitparameter $\mu$ verwendet:
\begin{equation}
   \label{eq:halbwertsbreite_formel_20}
    x\ua{1/2,f}= \SI{1.03\pm 0.01}{\kilo\eV}
\end{equation}
Die in \eqref{eq:halbwertsbreite_formel_20} aufgeführte
Halbwertsbreite gibt das bestmögliche Auflösungsvermögen
des Detektors an.

Das Comptonspektrum ist vergrößert in Abbildung \ref{fig:comptonspektrum}
dargestellt.
\begin{figure}
  \centering
  \includegraphics[width=0.7\textwidth]{../analysis/plots/caesium/compton_spektrum.pdf}
  \caption{Comptonspektrum der $^{137}{Cs}$ Quelle.}
  \label{fig:comptonspektrum}
\end{figure}
Mit eingezeichnet ist in der Abbildung der abgelesene Ort für den Rückstreupeak
und der Comptonkante:
\begin{align}
  \label{eq:rückstreu_kante_abgelesen}
  \begin{aligned}
    E\ua{rück}&=\num{480\pm 5}\,\map{Kanal} = \SI{191\pm4}{\kilo\eV}\\
    E\ua{kante}&=\num{1182 \pm 5}\,\map{Kanal} = \SI{473\pm4}{\kilo\eV}.
  \end{aligned}
\end{align}
Hierbei wurde eine Ableseungenauigkeit von $\pm\, 5\,\map{Kanal}$ angenommen.
Überprüft werden können die abgelesenen Wert $E\ua{rück}$ mit Gleichung \eqref{eq:rückstreu}.
Hierbei wird für den Rückstreupeak $\theta=\frac{\pi}{2}$ eingesetzt.
Der Wert für $E\ua{kante}$ kann mit Formel \eqref{eq: comptonkante_energie} geprüft werden.
\begin{align}
  \label{eq:rückstreu_kante_abgelesen}
  \begin{aligned}
    E\ua{rück, calc}&= \SI{184.2\pm0.1}{\kilo\eV}\\
    E\ua{kante, calc}&= \SI{475.9\pm0.1}{\kilo\eV}
  \end{aligned}
\end{align}
Die abgelesene und berechnete Energie der Comptonkante und des Rückstreupeaks besitzen
eine zur Nullmenge verschiedene Schnittmenge.

Als Nächstes soll der Inhalt des Comptonkontinums und des Vollenergiepeaks
verglichen werden. Hierzu wird jeweils über den Kanalinhalt des jeweiligen Spektrums
summiert:
\begin{equation}
  \label{eq:flacehe}
  I\ua{compton} = 62467 \, \map{Count}\cdot\map{Kanal}, \quad I\ua{peak}=13332\, \map{Count}\cdot\map{Kanal}
\end{equation}
Wird aus dem beiden der Quotient gebildet ist es möglich daraus eine Aussage über
die Wechselwirkungswahrscheinlichkeit zutreffen:
\begin{equation}
  \label{eq:qotient}
  \frac{I\ua{compton}}{I\ua{peak}} = 4.69
\end{equation}
Abschließend soll die Absoprtionswahrscheinlichkeit für den
Compton- und Photoeffekt bestimmt werden. Hierzu werden die
oben ermittelten Energien verwendet. Mit den Energien und Abbildung
\ref{fig:extinktionskoeffizient} ist es möglich den Extinktionskoeffizienten abzulesen
($\mu\ua{compton}=\num{0.35\pm0.05}$ und $\mu\ua{photo}=\num{0.06\pm{0.005}}$).
Werden diese in Gleichung
\begin{equation}
  \label{eq:Absorptionswahrscheinlichkeit}
  P\ua{i}=\left(1- \exp(-\mu_i d)\right) \cdot 100
\end{equation}
mit einer $d=\SI{3.9}{\centi\meter}$ eingesetzt ergibt sich:
\begin{align}
  \label{eq:absorptioswahrscheinlichkeit}
  \begin{aligned}
    P\ua{compton}&= \num{74\pm5}\,\% \\
    P\ua{Photo}&= \num{20.9\pm1.5}\, \%
    \frac{P\ua{compton}}{P\ua{Photo}}&=4.69
  \end{aligned}
\end{align}
Die berechnete Wechselwirkungswahrscheinlichkeit stimmt somit mit dem
Verhältnis der Flächen überein.
\FloatBarrier
\subsection{Untersuchung des ersten unbekannten Quelle}
\FloatBarrier
Das nach einer Messzeit von $t\ua{measurment}=\SI{2941}{\second}$ aufgenommene
Spektrum ist in Abbildung \ref{fig:spektrum_sb_or_ba} dargestellt.
\begin{figure}
  \centering
  \includegraphics[width=0.7\textwidth]{../analysis/plots/sb_or_ba/spectrum.pdf}
  \caption{Spektrum der ersten unbekannten Quelle.}
  \label{fig:spektrum_sb_or_ba}
\end{figure}
Mit eingezeichnet sind die mit \textsc{find\_peaks} abgelesenen Peaks.
Analog zu Kapitel \ref{sec:kalibrierung} werden die Energien der Peaks mit
einer Regressionsrechnung bestimmt.
\begin{table}
\centering
\caption{Regressionsparameter der Peak-Anpassung.}
\label{tab: results_peaks}
\begin{tabular}{S S[table-format=1.2]@{${}\pm{}$} S[table-format=1.2] S[table-format=1.2]@{${}\pm{}$} S[table-format=1.2] S[table-format=2.2]@{${}\pm{}$} S[table-format=1.2] }
\toprule
{$Kanal$} & \multicolumn{2}{c}{$A \:/\: \map{Zählung}$} & \multicolumn{2}{c}{$\sigma \:/\: \si{\kilo\eV}$} & \multicolumn{2}{c}{$\mu \:/\: \si{ \kilo\eV}$} \\ 
\midrule
2918 & 131.17 & 4.51 & 1.40 & 0.06 & 1172.50 & 0.06\\
3315 & 104.39 & 3.49 & 1.52 & 0.06 & 1331.35 & 0.06\\
\bottomrule
\end{tabular}
\end{table}

Durch verlgeich der Spektren von $^{125}\ce{Sb}$ (vgl. Quelle \cite{chartofnuclieds})
und $^{133}\ce{Ba}$ (vgl. Quelle \cite{chartofnuclieds}) mit dem
der Messung wird ersichtlich, dass es sich bei der unbekannten Quelle um $^{133}\ce{Ba}$
handelt. Aus der verwendeten Quelle können zusätzlich die Übergangswahrscheinlichkeiten $P\ua{über}$
entnommen werden.

Die Bestimmung der Aktivität $A\ua{k}$ der Barimum-Quelle erfolgt mit Gleichung \eqref{eq: zählrate}.
Die Aktivität wird für jeden Peak einzelent bestimmt und anschließend
gemittelt. Zunächst wird dazu die Vollenergienachweiseffizienz des Detektors $Q$ an jedem
Peak berechnet. Aufgelistet sind die Resultate in Tabelle \ref{tab: effizienz}.
\begin{table}
\centering
\caption{Berchente Vollenergienachweiseffizienz $^{133}\ce{Ba}$.}
\label{tab: effizienz}
\begin{tabular}{S S[table-format=2.2]@{${}\pm{}$} S[table-format=1.2] S[table-format=1.2]@{${}\pm{}$} S[table-format=1.2] }
\toprule
{$Channel$} & \multicolumn{2}{c}{$Energie \:/\: \si{ \kilo\eV}$} & \multicolumn{2}{c}{$Q$} \\ 
\midrule
208 & 83.38 & 0.31 & 0.52 & 0.08\\
693 & 277.80 & 1.02 & 0.30 & 0.06\\
758 & 303.85 & 1.11 & 0.28 & 0.06\\
890 & 356.76 & 1.31 & 0.24 & 0.06\\
959 & 384.42 & 1.41 & 0.22 & 0.06\\
\bottomrule
\end{tabular}
\end{table}

Die Ergebnisse für jeden Peak befinden sich in Tabelle \ref{tab: decay_rate_peak}.
\begin{table}
\centering
\caption{Bestimmte Aktivität für jeden Peak der $^{60}\ce{Co}$ Quelle.}
\label{tab: decay_rate_peak_co}
\begin{tabular}{S S[table-format=2.2]@{${}\pm{}$} S[table-format=1.2] S S S[table-format=1.2]@{${}\pm{}$} S[table-format=3.2] }
\toprule
{$Kanal$} & \multicolumn{2}{c}{$\mu \:/\: \si{ \kilo\eV}$} & {$P\ua{über}$} & {$\map{Fläche} \, / \, \map{Zählung} \cdot \map{Kanal} $} & \multicolumn{2}{c}{$A\ua{k} \:/\: \si{ \becquerel}$} \\
\midrule
2918 & 1172.50 & 0.06 & 0.9958 & 986 & 398.69 & 171.75\\
3315 & 1331.35 & 0.06 & 0.9998 & 778 & 363.00 & 163.55\\
\bottomrule
\end{tabular}
\end{table}

Gemittelt ergibt sich somit eine Aktivität von
\begin{equation}
  \label{eq:aktivität_Ba}
  \overline{A\ua{k}}=\SI{1.10\pm0.23 e3}{\becquerel}.
\end{equation}
\FloatBarrier
\subsection{Untersuchung des zweiten unbekannten Quelle}
Das nach einer Messzeit von $t\ua{measurment}=\SI{3412}{\second}$ aufgenommene
Spektrum ist in Abbildung \ref{fig:spektrum_sb_or_ba} dargestellt.
\begin{figure}
  \centering
  \includegraphics[width=0.7\textwidth]{../analysis/plots/unknown/spectrum_unknown.pdf}
  \caption{Spektrum der zweiten unbekannten Quelle.}
  \label{fig:spektrum_sb_or_ba}
\end{figure}
Mit eingezeichnet sind die mit \textsc{find\_peaks} abgelesenen Peaks.
Analog zu Kapitel \ref{sec:kalibrierung} werden die Energien der Peaks mit
einer Regressionsrechnung bestimmt.
\begin{table}
\centering
\caption{Regressionsparameter der Peak-Anpassung.}
\label{tab: results_peaks}
\begin{tabular}{S S[table-format=1.2]@{${}\pm{}$} S[table-format=1.2] S[table-format=1.2]@{${}\pm{}$} S[table-format=1.2] S[table-format=2.2]@{${}\pm{}$} S[table-format=1.2] }
\toprule
{$Kanal$} & \multicolumn{2}{c}{$A \:/\: \map{Zählung}$} & \multicolumn{2}{c}{$\sigma \:/\: \si{\kilo\eV}$} & \multicolumn{2}{c}{$\mu \:/\: \si{ \kilo\eV}$} \\ 
\midrule
2918 & 131.17 & 4.51 & 1.40 & 0.06 & 1172.50 & 0.06\\
3315 & 104.39 & 3.49 & 1.52 & 0.06 & 1331.35 & 0.06\\
\bottomrule
\end{tabular}
\end{table}

Durch den Vergleich des resultierenden Spektrums mit der Quelle
\cite{chartofnuclieds}, zeigt sich das es sich bei dem Strahler um $^{60}\ce{Co}$
handelt. Aus der Quelle \cite{chartofnuclieds} können auch die Übergangswahrscheinlichkeiten $P\ua{über}$
abgelesenen werden. Die berechnete Vollenergienachweiseffizienz $Q$ an
dem jeweiligen Peak befindet sich in Tabelle \ref{tab: effizienz_co}.
\begin{table}
\centering
\caption{Berchente Vollenergienachweiseffizienz $^{133}\ce{Ba}$.}
\label{tab: effizienz}
\begin{tabular}{S S[table-format=2.2]@{${}\pm{}$} S[table-format=1.2] S[table-format=1.2]@{${}\pm{}$} S[table-format=1.2] }
\toprule
{$Channel$} & \multicolumn{2}{c}{$Energie \:/\: \si{ \kilo\eV}$} & \multicolumn{2}{c}{$Q$} \\ 
\midrule
208 & 83.38 & 0.31 & 0.52 & 0.08\\
693 & 277.80 & 1.02 & 0.30 & 0.06\\
758 & 303.85 & 1.11 & 0.28 & 0.06\\
890 & 356.76 & 1.31 & 0.24 & 0.06\\
959 & 384.42 & 1.41 & 0.22 & 0.06\\
\bottomrule
\end{tabular}
\end{table}

Damit ist es möglich die Aktivität $A\ua{r}$ für jeden Peak zu bestimmen.
Der Inhalt eines Peaks wird durch Summierung der Kanalinhalte berechnet.
Gelistet sind die Aktivitäten $A\ua{r}$ in Tabelle \ref{tab: decay_rate_peak_co}.
\begin{table}
\centering
\caption{Bestimmte Aktivität für jeden Peak der $^{60}\ce{Co}$ Quelle.}
\label{tab: decay_rate_peak_co}
\begin{tabular}{S S[table-format=2.2]@{${}\pm{}$} S[table-format=1.2] S S S[table-format=1.2]@{${}\pm{}$} S[table-format=3.2] }
\toprule
{$Kanal$} & \multicolumn{2}{c}{$\mu \:/\: \si{ \kilo\eV}$} & {$P\ua{über}$} & {$\map{Fläche} \, / \, \map{Zählung} \cdot \map{Kanal} $} & \multicolumn{2}{c}{$A\ua{k} \:/\: \si{ \becquerel}$} \\
\midrule
2918 & 1172.50 & 0.06 & 0.9958 & 986 & 398.69 & 171.75\\
3315 & 1331.35 & 0.06 & 0.9998 & 778 & 363.00 & 163.55\\
\bottomrule
\end{tabular}
\end{table}

Als gemittelte Aktivität ergibt sich:
\begin{equation}
  \label{eq:gemittelt_aktivität_co}
  \ov{A}=\SI{450\pm200}{\becquerel}
\end{equation}
Auffällig ist der große Unsicherheitsbereich in Gleichung \eqref{eq:gemittelt_aktivität_co}.
