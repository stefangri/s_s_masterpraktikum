\section{Auswertung}
Im folgenden wird das $\gamma$-Spektrum vier verschiedner Proben untersucht.
Die erste Probe, \emph{Europium} $^{152}\ce{Eu}$, wird verwendet um den Detektor
zu kalibirieren. Hierzu werden die Channelskala in eine Energieskala transformiert.
Außerdem soll die energieabhängig Effizienz $Q$ des Detektor ermittelt werden.
Mit der zweiten Probe, \emph{Caesium} $^{137}\ce{Cs}$, sollen neben den
Detektoreigenschaften, der Vollenergiepeak und das Comptonspektrum untersucht werden.
Darüberhinaus soll die Aktivität $A$ der Caesium-Probe bestimmt werden.
Die Identität der letzten beiden Proben ist nicht bekannt und soll mit Hilfe
des Germaniumdetektors festgestellt werden. Zusätzlich soll die Aktivität
der unbekannten Probe berechnet werden.

Für die folgenden Anaylsen wird aus dem \textsc{python} Paktet \textsc{scipy.optimize}~\cite{scipy}
die Funktion \textsc{curve\_{fit}} für Regressionsberechnung verwendet.

\subsection{Kalibrierung des Germaniumdetektors}
Die Kalibrierung wird mit Hilfe einer Europium $^{152}\ce{Eu}$ Probe
durchgeführt. Das nach einer Messzeit von $t\ua{measurment}=\SI{3380}{\second}$
aufgenommene Energiespektrum ist in Abbildung \ref{fig:spectrum_europ} dargestellt.
\begin{figure}
  \centering
  \includegraphics[width=0.8\textwidth]{../analysis/plots/europium/spektrum_index.pdf}
  \caption{Gemessene Spektrum von Europium $^{152}\ce{Eu}$.}
  \label{fig:spectrum_europ}
\end{figure}
In der Abbildung \ref{fig:spectrum_europ} sind zusätzlich die mit
\textsc{scipy.signal.find\_{peaks}} ermittelten Peaks eingezeichnet.
An jeden Peak wird eine Gaußfuntkion
\begin{equation*}
  G(x)=A\exp{\left(-\frac{1}{2}\left(\frac{x-\mu}{\sigma}\right)^2\right)}
\end{equation*}
angepasst, um den Hochpunkt des Peaks gegeben durch den Parameter $\mu$ zu
bestimmen. Die Parameter befinden sich in Tabelle \ref{tab: results_peaks_eu}.
\input{../analysis/tabs/europium/peak_charakteristiken_eu_final.tex}
Die zu den Peaks gehörige Channelnummer wird mit der, aus der Anleitung \cite{anleitungV18}
bekannten, charakteristischen Energie zugeordnet.
Damit die gesamte Channelskala in eine Energieskala transformiert werden kann,
wird eine lineare Regression an die Funktion
\begin{equation*}
  g(x)=mx+b
\end{equation*}
durchgeführt. Zusätzlich wurde für die Regressionsrechnung der Datentuple\newline
$(Channelnummer~=~0,E_\gamma~=~0)$ hinzugefügt.
 Aus der Regressionsrechnung ergeben sich die folgenden Parameter
\begin{equation}
  \label{eq:channel_energie_trafo}
  m=\num{0.4009\pm0.0015}\,\frac{\si{\kilo\eV}}{\map{Channelnummer}}, \quad b=\SI{0.0\pm2.8}{\kilo\eV}.
\end{equation}
Aufällig ist der Fehler des Parameters $b$.
In der Abbildung \ref{fig:skalen_trafo_fit} ist der Fit mit den Fitpunkten dargestllt.
\begin{figure}
  \centering
  \includegraphics[width=0.8\textwidth]{../analysis/plots/europium/skalen_trafo_fit.pdf}
  \caption{Fitpunkte und Regressionsgerade der Energietransformation.}
  \label{fig:skalen_trafo_fit}
\end{figure}
Die aus der Transformation resusltierenden Energien sind in der Tabelle \ref{tabs: energy__peaks_eu'}
aufgelistet.
\begin{table}
\centering
\caption{Energiewerte der Peaks von $^{152}\ce{Eu}$.}
\label{tab:energy_peaks_eu}
\begin{tabular}{S S S[table-format=3.2]@{${}\pm{}$} S[table-format=1.2] }
\toprule
{$Kanal$} & {$E_{\gamma,theo}\: /\: \si{ \kilo\eV }$} & \multicolumn{2}{c}{$E_{\gamma} \:/\: \si{ \kilo\eV }$} \\
\midrule
309 & 121.78 & 121.91 & 3.48\\
614 & 244.70 & 244.49 & 3.60\\
861 & 344.30 & 343.76 & 3.74\\
1027 & 411.96 & 410.44 & 3.86\\
1108 & 443.96 & 443.12 & 3.93\\
1939 & 778.90 & 777.12 & 4.77\\
2158 & 867.37 & 865.21 & 5.04\\
2398 & 946.08 & 961.69 & 5.34\\
2701 & 1085.90 & 1083.28 & 5.75\\
2767 & 1112.10 & 1109.34 & 5.84\\
3501 & 1408.00 & 1404.70 & 6.89\\
\bottomrule
\end{tabular}
\end{table}

Die in diesem Abschnitt bestimmten Parameter werden in späteren Teilen der Auswertung
ohne weitere Anmerkung verwendet.

Der Detektor kann nicht jede Energie mit der selben Vollenergienachweiseffizienz
nachweisen, deshalb wird diese im Folgenden untersucht. Hierzu wird die Formel
\eqref{} nach $Q$ umgestellt. Die Größe $Z$ wird durch Summation der Channelinhalte
um ein Peak bestimmt. Dabei ist zu beachten, dass $Z$ durch die Messzeit $t_measurment$
dividiert wird. Die Übergangswahrscheinlichkeiten werden der Anleitung \cite{anleitungV18}
entnommen. Die Winkelverteilung wird mit Gleichung \eqref{} berechnet. Hierbei wurde
für den Abstand von Probe und Detektor der Wert $a=\SI{881}{\milli\meter}$ und für den
den Radius der Querschnittfläche des Detektors $r=\SI{27.5}{\milli\meter}$
verwendet. Damit folgt für die Winkelverteilung
\begin{equation}
  \eqref{eq:winkelverteilung}
  \frac{\Omega}{4\pi}=\num{0.0155}.
\end{equation}
Die anfängliche Aktivität der Probie lag nach Angaben der Anleitung \cite{anleitungV18}
am 01.10.2000 bei $A_0=\SI{4130\pm60}{\becquerel}$. Mit dem Zerfallsgesetzt kann die
Aktivität der Quelle am 29.10.2018 bestimmt werden:
\begin{equation}
  A(t\ua{29.10.2018})=A_0\exp\left(-\frac{t\ua{29.10.2018}\log(2)}{\tau\ua{1/2}}\right) = \SI{1636\pm24}{\becquerel}.
\end{equation}
Bei der Berechnung wurde die in der Anleitung \cite{anleitungV18} angegebene
Halbwertszeit von $^{152}\ce{Eu}$, $\tau\ua{1/2}=\SI{4943 \pm 5 }{\day}$ genutzt.
Eingetragen sind die berechneten Effizienzen der Probe in Tabelle \ref{tab: results_europium}.
\begin{table}
\centering
\caption{Bestimmten Energie und Effizienzwerte.}
\label{tab: results_europium}
\begin{tabular}{S S S S S[table-format=1.4]@{${}\pm{}$} S[table-format=1.4] }
\toprule
{$Kanal$} & {$E\ua{\gamma,lit} \, / \si{ \kilo\eV}$} & {$Z / \si{ \becquerel}$} & {$W$} & \multicolumn{2}{c}{$Q$} \\
\midrule
309 & 121.78 & 12078 & 0.286 & 0.4910 & 0.0071\\
614 & 244.70 & 2231 & 0.076 & 0.3413 & 0.0050\\
861 & 344.30 & 4848 & 0.265 & 0.2127 & 0.0031\\
1027 & 411.96 & 430 & 0.022 & 0.2273 & 0.0033\\
1108 & 443.96 & 495 & 0.031 & 0.1857 & 0.0027\\
1939 & 778.90 & 979 & 0.129 & 0.0882 & 0.0013\\
2158 & 867.37 & 325 & 0.042 & 0.0900 & 0.0013\\
2398 & 946.08 & 743 & 0.146 & 0.0592 & 0.0009\\
2701 & 1085.90 & 449 & 0.102 & 0.0512 & 0.0007\\
2767 & 1112.10 & 557 & 0.136 & 0.0476 & 0.0007\\
3501 & 1408.00 & 578 & 0.210 & 0.0320 & 0.0005\\
\bottomrule
\end{tabular}
\end{table}

Die nachfolgenden Regressionsrechnungen werden an Energien $E_\gamma>\SI{150}{\kilo\eV}$
durchgeführt.
An die bestimmten Vollenergienachweiseffizienzen wird eine Funtkion der Form
\begin{equation*}
  Q_1(E)=A_1*E^{B_1}
\end{equation*}
angepasst. Die Parameter folgen aus der Regressionsrechnung als:
\begin{equation}
\label{eq:effizienz_Q_1}
A_1 = \SI{2.1\pm1.1 e2}{\per\kilo\eV}, \quad B_2=\num{-1.17\pm 0.09}.
\end{equation}
Beim Vergleich der Parameter mit der Abschätzung für den Wirkungsquerschnitt
des Photoeffekts \eqref{} fällt auf das der Parmeter $B$ im Intervall im angegebenen
Intervall für $\delta$ liegt und $A_1/z_{\ce{Eu}}=\num{3.5\pm1.7}$ zumindest im
Fehlerbereich im Intervall des Parameters $\alpha$ liegt.
Außerdem wird an die Datenpunkte der Tabelle \ref{tab: results_europium}
eine Funktion der Form
\begin{equation*}
  Q_2(E)=A_2\exp\left(-B_2 E\right)+C
\end{equation*}
angepasst. Aus der Regressionsrechnung ergeben sich die folgenden Parameter
\begin{equation}
\label{eq:effizienz_Q_2}
A_2 = \num{0.64\pm0.09}, \quad B_2=\SI{32\pm6 e-4}{\per\kilo\eV}, \quad C = \num{0.030\pm0.016}.
\end{equation}
Dargestellt sind beide Fitfunktion und die Datenpunkte in Abbildung \ref{fig:effizienz}.
\begin{figure}
  \centering
  \includegraphics[width=0.8\textwidth]{../analysis/plots/europium/effizienz.pdf}
  \caption{Ermittelte Vollenergienachweiseffizienz $Q$ des Detektors.}
  \label{fig:effizienz}
\end{figure}
Für die späteren Aktivitätsbestimmung wird die Effizienzfunktion $Q_2$ mit den
ermittelten Parametern ohne weitere Kommentare verwendet.
