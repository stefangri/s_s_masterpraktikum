\section{Versuchsaufbau/-durchführung}
Nachfolgend soll nun genauer die Funktionsweise eines Germaniumdetektors erklärt werden. Hierbei wird insbesondere 
die elektrische Schaltung vorgestellt. Anschließend werden die durchgeführten Messungen motiviert.

Das Germanium-Modul verfügt über eine zylindrische Geometrie. Von Außen ist eine stark mit Lithium dotierte $n$
Schicht aufgebracht, von Innen eine mit Gold dotierte $p$ Schicht. An die Schichten sind Elektroden angeschlossen, zwischen denen 
die notwendige Hochspannung von $\SI{}{\volt}$ angelegt ist. Die Probe wird über dem Zylinder in eine Schraubvorrichtung 
befestigt. Probe und Detektor befinden sich innerhalb eines Bleikastens, der insbesondere den Einfluss von kosmischer 
Strahlung minimieren soll. Zudem schützt eine innere dünne Kupferschicht vor $\alpha$-Strahlung, die von der Bleiummantelung ausgehen 
könnte. 

Der am Halbleiterelement erzeugt Stromimpuls muss zunächst in einen Spannungsimpuls umgewandelt werden. Dies geschieht mit 
dem Vorverstärker in Abbildung~\ref{}. Mithilfe eines kapazitiv rückgekoppelten Operationsverstärkers wird durch elektrische 
Integration ein Spannungspegel erzeugt, der proportional zur Photonenenergie seien soll. Um einen zeitlich 
begrenzten Spannungsimpuls zu erhalten muss der Kondensator $C_K$ nach jedem Nachweis zu entladen, was durch die 
Anordnung in abbildung~\ref{} erfolgt. Durch eine LED nach jedem Impuls die Gate-Drain-Schicht des 
Eingangs-Feldeffekttransistors (FET) leitend gemacht und so kann die Ladung von $C_K$ abfließen.

Das vorverstärkte Signal gelangt zu einem Hauptverstärker, der die Spannungsimpulse abschließend verstärkt, sodass 
sie Vielkanalanalysator in Kanaäle eingeordnet werden können. Dies geschieht über eine Schaltung, wie sie in Abbildung~\ref{}
dargestellt ist. Die Messung der Impulshöhe basiert hier auf einer Zeitmessung. Der Spannungsimpuls lädt einen 
Kondensator auf, dessen Abklingszeit bei Entladung von der Amplitude des Pulses abhängt. Ein Und-Gatter, verknüpft 
das zu untersuchende Signal mit einem Quarzoszillator. Die Zählrate am Binärzähler ist damit ein direktes Maß für 
die Impulshöhe und somit auch für die Energie der Photonen. 
Die Zählraten werden abschließend in einem Computer verarbeitet und als Ergebnis wird ein Histogramm erzeugt, auf dessen 
$x$-Achse Kanalnummern sind und auf der $y$-Achse Zählraten. Ein letzte wichtige Größe zur Charakterisierung des Detektors ist 
die Effizienz $Q$

Zunächst muss herausgefunden werden, welcher Zusammenhang zwischen Kanalnummer und Energie, sowie zwischen 

