\newpage
\section{Auswertung}
Im Folgenden werden die gewonnenen Messergebnisse vorgestellt. Die notwendigen Ausgleichsrechungen werden 
mit Hilfe der \emph{python} Bibliothek \emph{scipy.optimize} durchgeführt.

\subsection{Kontrastmessung}
Die gemessenen maximalen und minimalen Diodenspannungen unter variablem Winkel des Polarisationsfilters sind in 
Tabelle~\ref{tab: messwerte_kontrast} eingetragen.
Die Daten werden, wie in der Abschnitt \ref{sec: procedure} motiviert, an eine Funktion der Gestalt
\begin{equation}
    K(\vartheta) = A \left|\sin \left(2 \vartheta + \vartheta_0 \right)\right| + B 
    \label{eq: model_contrast}
\end{equation}
angepasst. Für die Fitparameter ergeben sich die folgenden Werte:
\begin{align}
    \begin{aligned}
    A &= \input{../analysis/results/result_A.tex} \quad  & \vartheta_0 &= \input{../analysis/results/result_vartheta_0.tex} = \SI{-0.092(21)}{\radian} \\ 
    C &= \input{../analysis/results/result_C.tex}.
    \end{aligned}
\end{align}
Eine Darstellung des Fits und der Messwerte ist in Abbildung~\ref{fig: kontrast_fit} einzusehen.
Die nachfolgenden Messungen sind mit der Einstellung $\vartheta = \SI{55}{\degree}$ durchgeführt worden, da hier der maximale 
Kontrast von ca. $\SI{96}{\percent}$ gemessen wurde.  

\begin{table} 
\centering 
\caption{Winkel des Polarisationsfilter, sowie maximal und minimal gemessene Spannung zur Bestimmung des Kontrast des Interferometers.} 
\label{tab: messwerte_kontrast} 
\begin{tabular}{S S S S } 
\toprule  
{$\text{Winkel} / \si{ \degree}$} & {$U_{max} / \si{ \volt}$} & {$U_{min} / \si{ \volt}$} & {$\text{Kontrast} / \si{ \percent}$} \\ 
\midrule  
0 & 4.24 & 3.44 & 10.4\\ 
10 & 3.95 & 2.20 & 28.4\\ 
20 & 3.47 & 1.20 & 48.5\\ 
30 & 3.08 & 0.56 & 69.1\\ 
35 & 3.03 & 0.36 & 78.8\\ 
40 & 3.03 & 0.27 & 83.9\\ 
45 & 3.22 & 0.17 & 89.9\\ 
50 & 3.53 & 0.14 & 92.4\\ 
55 & 4.06 & 0.09 & 95.5\\ 
60 & 4.12 & 0.19 & 91.3\\ 
70 & 4.49 & 0.69 & 73.4\\ 
80 & 4.91 & 1.81 & 46.1\\ 
90 & 4.86 & 4.08 & 8.7\\ 
100 & 6.59 & 3.77 & 27.3\\ 
110 & 9.31 & 2.31 & 60.2\\ 
120 & 11.38 & 1.06 & 82.9\\ 
130 & 11.50 & 0.56 & 90.7\\ 
140 & 11.50 & 0.56 & 90.7\\ 
150 & 11.12 & 1.56 & 75.4\\ 
160 & 9.06 & 2.31 & 59.3\\ 
170 & 6.81 & 3.19 & 36.2\\ 
180 & 4.81 & 3.25 & 19.4\\ 
\bottomrule 
\end{tabular} 
\end{table}

\begin{figure}
    \centering
    \includegraphics[scale = 1]{../analysis/results/kontrast_fit.pdf}
    \caption{Darstellung der Messwerte und des Fits zur Untersuchung des Kontrasts in Abhängigkeit vom eingestellten Winkel des Polarisationsfilters.}
    \label{fig: kontrast_fit}
\end{figure}


\subsection{Messung des Brechungsindex von Glas}
In Tabelle~\ref{tab: counts_glas} sin die Messwerte für die Anzahl der $2\pi$-Phasenverschiebungen unter Drehung der Glasplatten um $\SI{10}{\degree}$
eingetragen. Aus diesen können mittels Gleichung~\eqref{tab: counts_glas} 
(mit $\Delta\vartheta = \vartheta_0 = \SI{10}{\degree}$) jeweils die Brechungsindices berechnet werden. Diese sind ebenfalls in Tabelle~\ref{tab: counts_glas}  
aufgeführt. Als Mittelwert ergibt sich 
\begin{equation}
    n_{\text{Glas}} = \input{../analysis/results/result_n_glas.tex}.
\end{equation}

\begin{table} 
\centering 
\caption{Gemessene Anzahl der $2\pi$ Phasenverschiebungen (Counts) unter Drehung der Glasplatten um $\SI{10}{\degree}$, sowie daraus berechnete Brechungsindices $n$.} 
\label{tab: counts_glas} 
\begin{tabular}{S S } 
\toprule  
{$\text{Counts} / \si{ }$} & {$n / \si{ }$} \\ 
\midrule  
39 & 1.37\\ 
39 & 1.37\\ 
37 & 1.34\\ 
34 & 1.31\\ 
35 & 1.32\\ 
37 & 1.34\\ 
37 & 1.34\\ 
33 & 1.30\\ 
36 & 1.33\\ 
34 & 1.31\\ 
\bottomrule 
\end{tabular} 
\end{table}

\subsection{Druckabhängigkeit des Brechungsindex von Luft}
Die Ergebnisse der drei Messungen zur Untersuchung des Zusammenhangs zwischen dem Brechungsindex von Luft und dem Druck sind in 
Tabelle~\ref{tab: messwerte_n_gas} eingetragen. Es ist ersichtlich, dass die Messwerte der drei Reihen fast übereinstimmen, weswegen eine grafische
Darstellung nur exemplarisch für eine Messreihe in Abbildung~\ref{fig: fit_druck} erfolgt. Für alle Messreihen analog werden aus 
den $2\pi$-Phasenverschiebungen gemäß~\eqref{eq:phase_shit_gases} die Brechungsindices bestimmt und die Daten $(n, p)$ für eine Regression an 
das lineare Modell
\begin{equation}
    n = A p + B
\end{equation}
verwendet. Für die Parameter $A$ und $B$ ergeben sich die in Tabelle~\ref{tab: fitparams_druck} aufgeführten Werte. Hier ist ebenfalls eine 
Berechnung für den Brechungsindex bei $\SI{1}{\bar}$ angegeben. Als ungewichteter Mittelwert ergibt sich
\begin{equation}
    \overline{n}\left(\SI{1}{\bar} \right) = \input{../analysis/results/result_n.tex},
\end{equation}
welcher im folgenden Abschnitt durch den Vergleich mit einem Literaturwert
diskutiert werden kann. 

\begin{figure}
    \centering
    \includegraphics[scale = 1]{../analysis/results/druck_fit_1.pdf}
    \caption{Grafische Darstellung der Messwerte der ersten Messreihe zur Bestimmung des Zusammenhangs zwischen 
    Druck $p$ und Brechungsindex $n$. Zusätzlich ist eine Regressionsgerade eingezeichnet, deren Parameter in 
    Tabelle~\ref{tab: fitparams_druck} eingetragen sind.}
    \label{fig: fit_druck}
\end{figure}


\begin{table} 
\centering 
\caption{Messwerte der drei Reihen zur Bestimmung des Brechungsindex von Luft. Hierhi bezeichnet $p_i$ den gemessenen Druck und $\text{Counts}_i$ die gezählte Anzahl an $2\pi$ Phasenverschiebungen.} 
\label{tab: messwerte_n_gas} 
\begin{tabular}{S S S S S S } 
\toprule  
{$p_1 / \si{ \milli\bar}$} & {$\text{Counts}_1$} & {$p_2 / \si{ \milli\bar}$} & {$\text{Counts}_2$} & {$p_3 / \si{ \milli\bar}$} & {$\text{Counts}_3 $} \\ 
\midrule  
9 & 0 & 8 & 0 & 8 & 0\\ 
100 & 4 & 100 & 4 & 100 & 4\\ 
200 & 8 & 200 & 8 & 200 & 9\\ 
300 & 12 & 300 & 13 & 300 & 13\\ 
400 & 16 & 400 & 17 & 400 & 17\\ 
500 & 21 & 500 & 21 & 500 & 21\\ 
600 & 25 & 600 & 25 & 600 & 25\\ 
700 & 29 & 700 & 29 & 700 & 29\\ 
800 & 33 & 800 & 33 & 800 & 33\\ 
900 & 37 & 900 & 37 & 900 & 37\\ 
996 & 41 & 996 & 41 & 996 & 41\\ 
\bottomrule 
\end{tabular} 
\end{table}

\begin{table} 
\centering 
\caption{Berechnete Brechungsindices von Luft aus den Werten der Tabelle~\ref{tab: messwerte_n_gas}.} 
\label{tab: berechnete_n_gas} 
\begin{tabular}{S S[table-format=1.8]@{${}\pm{}$} S[table-format=1.8] S[table-format=1.8]@{${}\pm{}$} S[table-format=1.8] S[table-format=1.8]@{${}\pm{}$} S[table-format=1.8] } 
\toprule  
{$p / \si{\milli\bar}$} & \multicolumn{2}{c}{$n_1$} & \multicolumn{2}{c}{$n_2$} & \multicolumn{2}{c}{$n_3$} \\ 
\midrule  
100 & 1.00001266 & 0.00000001 & 1.00001266 & 0.00000001 & 1.00001266 & 0.00000001\\ 
200 & 1.00002532 & 0.00000003 & 1.00002532 & 0.00000003 & 1.00002848 & 0.00000003\\ 
300 & 1.00003798 & 0.00000004 & 1.00004114 & 0.00000004 & 1.00004114 & 0.00000004\\ 
400 & 1.00005064 & 0.00000005 & 1.00005380 & 0.00000005 & 1.00005380 & 0.00000005\\ 
500 & 1.00006646 & 0.00000007 & 1.00006646 & 0.00000007 & 1.00006646 & 0.00000007\\ 
600 & 1.00007912 & 0.00000008 & 1.00007912 & 0.00000008 & 1.00007912 & 0.00000008\\ 
700 & 1.00009178 & 0.00000009 & 1.00009178 & 0.00000009 & 1.00009178 & 0.00000009\\ 
800 & 1.00010444 & 0.00000010 & 1.00010444 & 0.00000010 & 1.00010444 & 0.00000010\\ 
900 & 1.00011710 & 0.00000012 & 1.00011710 & 0.00000012 & 1.00011710 & 0.00000012\\ 
996 & 1.00012976 & 0.00000013 & 1.00012976 & 0.00000013 & 1.00012976 & 0.00000013\\ 
\bottomrule 
\end{tabular} 
\end{table}




\begin{table} 
\centering 
\caption{Ermittelte Regressionsparameter $A$ und $B$ der Messung zur Bestimmung der Abhängigkeit zwischen Brechungsindex $n$ und Druck p. Zudem ist jeweils der Wert $n_{\mathup{norm}}$ zum Vergleich mit der Literatur angegeben.} 
\label{tab: fitparams_druck} 
\begin{tabular}{S S[table-format=1.2]@{${}\pm{}$} S[table-format=1.2] S[table-format=1.7]@{${}\pm{}$} S[table-format=1.7] S[table-format=1.7]@{${}\pm{}$} S[table-format=1.7] } 
\toprule  
{$\text{Messung}$} & \multicolumn{2}{c}{$A \:/\: \si{ 10^{-7}\milli\bar^-2}$} & \multicolumn{2}{c}{$B$} & \multicolumn{2}{c}{$n_\mathup{norm}$} \\ 
\midrule  
1 & 5.27 & 0.03 & 0.9999965 & 0.000002 & 1.0002616 & 0.000002\\ 
2 & 5.23 & 0.04 & 1.0000007 & 0.000002 & 1.0002619 & 0.000002\\ 
3 & 5.20 & 0.04 & 1.0000036 & 0.000003 & 1.0002616 & 0.000003\\ 
\bottomrule 
\end{tabular} 
\end{table}






