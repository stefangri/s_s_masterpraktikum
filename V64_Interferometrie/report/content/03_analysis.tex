\section{Auswertung}


\subsection{Kontrastmessung}


\begin{figure}
    \centering
    \includegraphics[scale = 1]{../analysis/results/kontrast_fit.pdf}
\end{figure}

Die Daten werden, wie in der Abschnitt \ref{} motiviert, an eine Funktion der Gestalt
\begin{equation}
    K(\vartheta) = A \sin \left(2 \vartheta + \vartheta_0 \right) + c 
\end{equation}
angepasst. Für die Fitparameter ergeben sich die folgenden Werte:
\begin{align}
    \begin{aligned}
    \input{../analysis/results/result_A.tex} \quad  & \input{../analysis/results/result_vartheta_0.tex} = \SI{-0.092(21)}{\radian} \\ 
    \input{../analysis/results/result_C.tex}.
    \end{aligned}
\end{align}



\subsection{Messung des Brechungsindex von Glas}




\subsection{Druckabhängigkeit des Brechungsindex von Luft}


