\section{Auswertung}


\subsection{Kontrastmessung}


\begin{figure}
    \centering
    \includegraphics[scale = 1]{../analysis/results/kontrast_fit.pdf}
\end{figure}

Die Daten werden, wie in der Abschnitt \ref{} motiviert, an eine Funktion der Gestalt
\begin{equation}
    K(\vartheta) = A \sin \left(2 \vartheta + \vartheta_0 \right) + c 
\end{equation}
angepasst. Für die Fitparameter ergeben sich die folgenden Werte:
\begin{align}
    \begin{aligned}
    A &= \input{../analysis/results/result_A.tex} \quad  & \vartheta_0 &= \input{../analysis/results/result_vartheta_0.tex} = \SI{-0.092(21)}{\radian} \\ 
    C &= \input{../analysis/results/result_C.tex}.
    \end{aligned}
\end{align}



\subsection{Messung des Brechungsindex von Glas}




\subsection{Druckabhängigkeit des Brechungsindex von Luft}


\begin{figure}
    \centering
    \includegraphics[scale = 1]{../analysis/results/druck_fit_1.pdf}
    \label{fig: fit_druck}
    \caption{Grafische Darstellung der Messwerte der ersten Messreige zur Bestimmung des Zusammenhangs zwischen 
    Druck $p$ und Brechungsindex $n$. Zusätzlich ist eine Regressionsgerade eingezeichnet, deren Parameter in 
    Tabelle~\ref{tab: fitparams_druck} eingetragen sind.}
\end{figure}

\begin{table} 
\centering 
\caption{Ermittelte Regressionsparameter $A$ und $B$ der Messung zur Bestimmung der Abhängigkeit zwischen Brechungsindex $n$ und Druck p. Zudem ist jeweils der Wert $n_{\mathup{norm}}$ zum Vergleich mit der Literatur angegeben.} 
\label{tab: fitparams_druck} 
\begin{tabular}{S S[table-format=1.2]@{${}\pm{}$} S[table-format=1.2] S[table-format=1.7]@{${}\pm{}$} S[table-format=1.7] S[table-format=1.7]@{${}\pm{}$} S[table-format=1.7] } 
\toprule  
{$Messung / \si{ }$} & \multicolumn{2}{c}{$A \:/\: \si{ 10^{-7}\milli\bar^-1}$} & \multicolumn{2}{c}{$B \:/\: \si{ }$} & \multicolumn{2}{c}{$n_\mathup{norm} \:/\: \si{ }$} \\ 
\midrule  
1 & 1.32 & 0.01 & 0.9999991 & 0.0000005 & 1.0001308 & 0.0000009\\ 
2 & 1.31 & 0.01 & 1.0000002 & 0.0000006 & 1.0001309 & 0.0000012\\ 
3 & 1.30 & 0.01 & 1.0000009 & 0.0000006 & 1.0001308 & 0.0000013\\ 
\bottomrule 
\end{tabular} 
\end{table}


\begin{align}
    \overline{n}\left(\SI{1}{\bar} \right) = \input{../analysis/results/result_n.tex}
\end{align}


