\section{Auswertung}


\subsection{Kontrastmessung}
Die gemessenen maximal- und minimal Diodenspannungen unter variablem Winkel des Polarisationswinkels sind in 
Tabelle~\ref{tab: messwerte_kontrast} eingetragen.

\begin{table} 
\centering 
\caption{Winkel des Polarisationsfilter, sowie maximal und minimal gemessene Spannung zur Bestimmung des Kontrast des Interferometers.} 
\label{tab: messwerte_kontrast} 
\begin{tabular}{S S S S } 
\toprule  
{$\text{Winkel} / \si{ \degree}$} & {$U_{max} / \si{ \volt}$} & {$U_{min} / \si{ \volt}$} & {$\text{Kontrast} / \si{ \percent}$} \\ 
\midrule  
0 & 4.24 & 3.44 & 10.4\\ 
10 & 3.95 & 2.20 & 28.4\\ 
20 & 3.47 & 1.20 & 48.5\\ 
30 & 3.08 & 0.56 & 69.1\\ 
35 & 3.03 & 0.36 & 78.8\\ 
40 & 3.03 & 0.27 & 83.9\\ 
45 & 3.22 & 0.17 & 89.9\\ 
50 & 3.53 & 0.14 & 92.4\\ 
55 & 4.06 & 0.09 & 95.5\\ 
60 & 4.12 & 0.19 & 91.3\\ 
70 & 4.49 & 0.69 & 73.4\\ 
80 & 4.91 & 1.81 & 46.1\\ 
90 & 4.86 & 4.08 & 8.7\\ 
100 & 6.59 & 3.77 & 27.3\\ 
110 & 9.31 & 2.31 & 60.2\\ 
120 & 11.38 & 1.06 & 82.9\\ 
130 & 11.50 & 0.56 & 90.7\\ 
140 & 11.50 & 0.56 & 90.7\\ 
150 & 11.12 & 1.56 & 75.4\\ 
160 & 9.06 & 2.31 & 59.3\\ 
170 & 6.81 & 3.19 & 36.2\\ 
180 & 4.81 & 3.25 & 19.4\\ 
\bottomrule 
\end{tabular} 
\end{table}


\begin{figure}
    \centering
    \includegraphics[scale = 1]{../analysis/results/kontrast_fit.pdf}
    \label{fig: kontrast_fit}
    \caption{Darstellung der Messwerte und des Fits zur Untersuchung des Kontrasts in Abhängigkeit vom eingestellten Winkel des Polarisationsfilters.}
\end{figure}

Die Daten werden, wie in der Abschnitt \ref{} motiviert, an eine Funktion der Gestalt
\begin{equation}
    K(\vartheta) = A \left|\sin \left(2 \vartheta + \vartheta_0 \right)\right| + B 
\end{equation}
angepasst. Für die Fitparameter ergeben sich die folgenden Werte:
\begin{align}
    \begin{aligned}
    A &= \input{../analysis/results/result_A.tex} \quad  & \vartheta_0 &= \input{../analysis/results/result_vartheta_0.tex} = \SI{-0.092(21)}{\radian} \\ 
    C &= \input{../analysis/results/result_C.tex}.
    \end{aligned}
\end{align}
Eine Darstellung des Fits und der Messwerte ist in Abbildung~\ref{fig: kontrast_fit} einzusehen.
Die nachfolgenden Messungen sind mit der Einstellung $\vartheta = \SI{55}{\degree}$ durchgeführt worden, da hier der maximale 
Kontrast von ca. $\SI{96}{\percent}$ gemessen wurde.  



\subsection{Messung des Brechungsindex von Glas}


\begin{table} 
\centering 
\caption{Gemessene Anzahl der $2\pi$ Phasenverschiebungen (Counts) unter Drehung der Glasplatten um $\SI{10}{\degree}$, sowie daraus berechnete Brechungsindices $n$.} 
\label{tab: counts_glas} 
\begin{tabular}{S S } 
\toprule  
{$\text{Counts}$} & {$n$} \\ 
\midrule  
39 & 1.37\\ 
39 & 1.37\\ 
37 & 1.34\\ 
34 & 1.31\\ 
35 & 1.32\\ 
37 & 1.34\\ 
37 & 1.34\\ 
33 & 1.30\\ 
36 & 1.33\\ 
34 & 1.31\\ 
\bottomrule 
\end{tabular} 
\end{table}


\begin{equation}
    n_{\text{Glas}} = \input{../analysis/results/result_n_glas.tex}
\end{equation}




\subsection{Druckabhängigkeit des Brechungsindex von Luft}


\begin{figure}
    \centering
    \includegraphics[scale = 1]{../analysis/results/druck_fit_1.pdf}
    \label{fig: fit_druck}
    \caption{Grafische Darstellung der Messwerte der ersten Messreige zur Bestimmung des Zusammenhangs zwischen 
    Druck $p$ und Brechungsindex $n$. Zusätzlich ist eine Regressionsgerade eingezeichnet, deren Parameter in 
    Tabelle~\ref{tab: fitparams_druck} eingetragen sind.}
\end{figure}

\begin{table} 
\centering 
\caption{Ermittelte Regressionsparameter $A$ und $B$ der Messung zur Bestimmung der Abhängigkeit zwischen Brechungsindex $n$ und Druck p. Zudem ist jeweils der Wert $n_{\mathup{norm}}$ zum Vergleich mit der Literatur angegeben.} 
\label{tab: fitparams_druck} 
\begin{tabular}{S S[table-format=1.2]@{${}\pm{}$} S[table-format=1.2] S[table-format=1.7]@{${}\pm{}$} S[table-format=1.7] S[table-format=1.7]@{${}\pm{}$} S[table-format=1.7] } 
\toprule  
{$Messung / \si{ }$} & \multicolumn{2}{c}{$A \:/\: \si{ 10^{-7}\milli\bar^-1}$} & \multicolumn{2}{c}{$B \:/\: \si{ }$} & \multicolumn{2}{c}{$n_\mathup{norm} \:/\: \si{ }$} \\ 
\midrule  
1 & 1.32 & 0.01 & 0.9999991 & 0.0000005 & 1.0001308 & 0.0000009\\ 
2 & 1.31 & 0.01 & 1.0000002 & 0.0000006 & 1.0001309 & 0.0000012\\ 
3 & 1.30 & 0.01 & 1.0000009 & 0.0000006 & 1.0001308 & 0.0000013\\ 
\bottomrule 
\end{tabular} 
\end{table}


\begin{align}
    \overline{n}\left(\SI{1}{\bar} \right) = \input{../analysis/results/result_n.tex}
\end{align}


