\setcounter{page}{1}
\section*{Zielsetzung}
In dem Versuch $V64$ wird mittels eines \emph{Sagnac Interferometer}
der Brechungsindex $n$ von Luft und einer Glasplatte bestimmt.
Hierbei bietet ein Sagnac Interferometer den Vorteil, dass sich
die beiden Strahlen des Interfeometer die selbe Strecke zurücklegen.
Hierdurch ist das Sagnac Interfeometer stabiler gegen externe Störung
als bspw. ein Michelson Interfeometer.

\section{Theorie}
Der Brechungsindex $n$ wird in mit einem Interferometer mit Hilfe von
Interferenzeffekten bestimmt. Interferent entstehen durch
eine Phasendifferenz $\Delta\phi$ zwischen zwei kohärenten Lichtstrahlen.
Eine solche Phasendifferenz wird in Gasen und Festkörpern unterschiedlich erzeugt.

%Brechungsindex von Gasen
%Brechungsindex von Festkörpern(Glasen)
%Polarisation
\subsection{Brechungsindex von Gasen}
Propagiert ein Lichtrahl von Medium $A$ durch eine Gasezelle der Länge $L$ mit Medium $B$,
so führt das zu einer Phasendifferenz $\Delta\phi$ relativ gesehen zu einem Lichtstrahl der sich
nur durch Medium $A$ bewegt. Erzeugt wird diese Phasendifferenz durch die Änderung der
Phasengeschwindigkeit im Medium $v\ua{ph}=\frac{\map{c}}{n_B}=\frac{\omega}{k}$, wobei $\map{c}$
die Vakuumlichtgeschwindigkeit repräsentiert. Eine Änderung der Phasengeschwindigkeit
verursacht eine Änderung der Wellenzahl $k$ und somit eine Phasenverschiebung
\begin{equation}
  \label{eq:phase_shit_gases}
\Delta\phi = kL = \frac{2\pi}{\lambda\ua{vac}}(n_B-n_A)L,
\end{equation}
mit der Vakuumwellenlänge $\lambda\ua{vac}$ und den Brechungsindizies der
Medien $A$ und $B$.
Zu beachten ist, dass der Brechungsindex eines Gases allgemein von dem Druck abhängt.
Die Abhängigkeit wird in einem Interferometer verwendet,um zwei kohärente Strahlen
interfiereren zu lassen. Hierbei hängt die Art der Interfernz von der Phasenverschiebung ab:
\begin{align}
  \Delta\phi &=0,2\pi,4\pi,\dots \qquad \text{Konstruktive Interfernz} \label{eq:Konstruktive}\\
  \Delta\phi &= \pi, 3\pi, 5\pi,\dots \qquad \text{Destruktive Interferenz}. \label{eq:Destruktive}
\end{align}
Durch eine Beobachtung bzw. Zählung der Interferenzmaxima $M$ bei einer gegebenen Druckdifferenz
ist es möglich den Brechungsindex zu bestimmen.
\textbf{Lorentz-Lorenz Kurve}

\subesction{Brechungsindex von Festkörpern}
Im Gegensatz zu Gasen ist der Brechungsindex eines Festkörpers nicht Druckabhaängig.
Deshalb muss hier eine andere Methodik verwendet werden
