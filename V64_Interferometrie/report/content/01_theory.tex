\setcounter{page}{1}
\section*{Zielsetzung}
In dem Versuch $V64$ wird mittels eines \emph{Sagnac Interferometer}
der Brechungsindex $n$ von Luft und einer Glasplatte bestimmt.
Hierbei bietet ein Sagnac Interferometer den Vorteil, einer
besseren Stabilität gegen externe Störung als zum Beispiel
ein Michelson Interfermoter. Geschuldet ist die höhere Stabilität
dem Fakt, dass die Lichtstrahlen des Interfermoeters den selben Weg zurück legen.

\section{Theorie}
Der Brechungsindex $n$ wird mit einem Interferometer mit Hilfe von
Interferenzeffekten bestimmt. Interferenzen entstehen durch
eine Phasendifferenz $\Delta\phi$ zwischen zwei kohärenten Lichtstrahlen.
Eine solche Phasendifferenz wird in Gasen und Festkörpern unterschiedlich erzeugt.
Bevor auf diese Problemstellung eingeangen wird, werden zunächst Interferenzen
mit dem Wellencharakter von Licht motviert.
%Brechungsindex von Gasen
%Brechungsindex von Festkörpern(Glasen)
%Polarisation
\subsection{Beschreibung von Licht}

Licht kann als elektromagnetische Welle beschrieben werden.
Auf Grund der Kopplung des elektrischen und magnetischen Teil, reicht
es für eine vollständige Beschreibung von Licht aus nur einen der beiden Teile zu betrachten.
Im Folgenden betrachten wir deshalb den elektrischen Teil.
Für monochromatisches Licht kann das elektrische Feld $\vec{E}$ mathematisch beschrieben
werden durch
\begin{equation*}
  \vec{E} = \vec{E_0} \map{exp}\left(i (\omega t-\vec{k}\vec{r})\right),
\end{equation*}
hierbei repräsentiert $\vec{E_0}$ die Polarisation des Feldes.
Die Überlagerung von zwei Wellen $\vec{E}_1$ und $\vec{E}_2$ kann es zu Interferenzeffekten führen.
Hierbei gilt im allgemein für die Intensität der überlagerten Wellen:
\begin{align}
  \label{eq:Interferenz_enstehung}
  \begin{aligned}
  \vec{E}_1&=\vec{E}_{0,1} \map{exp}\left(i (\omega t-\vec{k}\vec{r})\right), \qquad \vec{E}_2 =\vec{E}_{0,2} \map{exp}\left(i (\omega t-\vec{k}\vec{r}+\Delta\phi)\right) \\
  I &\propto \left<\be{E_1+E_2}^2\right> = E_{0,1}^2 + E_{0,2}^2 + 2E_{0,1}E_{0,2}\cos(\delta)\cos(\Delta\phi),
\end{aligned}
\end{align}
wobei $\Delta\phi$ die Phasendifferenz zwischen $\vec{E_1}$ und $\vec{E_2}$ darstellt und
$\delta$ den Polarisationswinkel der beiden Felder repräsentiert.
Die Art der Interfernz hängt somit von der Phasendifferenz ab:
\begin{align}
  \Delta\phi &=0,2\pi,4\pi,\dots \qquad \text{Konstruktive Interfernz} \label{eq:Konstruktive}\\
  \Delta\phi &= \pi, 3\pi, 5\pi,\dots \qquad \text{Destruktive Interferenz}. \label{eq:Destruktive}
\end{align}
Anzumerken ist, dass senkrecht zu einander polarisierts Licht $\delta=\frac{\pi}{2}$ nicht
interferenzfähig ist.
\subsection{Brechungsindex von Gasen}
Propagiert ein Lichtrahl von Medium $A$ durch eine Gasezelle der Länge $L$ mit Medium $B$,
so führt das zu einer Phasendifferenz $\Delta\phi$ relativ gesehen zu einem Lichtstrahl der sich
nur durch Medium $A$ bewegt. Erzeugt wird diese Phasendifferenz durch die Änderung der
Phasengeschwindigkeit im Medium $v\ua{ph}=\frac{\map{c}}{n_B}=\frac{\omega}{k}$, wobei $\map{c}$
die Vakuumlichtgeschwindigkeit darstellt. Eine Änderung der Phasengeschwindigkeit
verursacht eine Änderung der Wellenzahl $k$ und somit eine Phasenverschiebung
\begin{equation}
  \label{eq:phase_shit_gases}
\Delta\phi = kL = \frac{2\pi}{\lambda\ua{vac}}(n_B-n_A)L,
\end{equation}
mit der Vakuumwellenlänge $\lambda\ua{vac}$ und den Brechungsindizies der
Medien $A$ und $B$.
Zu beachten ist, dass der Brechungsindex eines Gases allgemein von dem Druck $p$ abhängt.
Beschrieben wird diese Druckabhängigkeit durch das Lorentz-Lorenz Gesetz.
Im Rahmen dieses Protokolls wird eine genährte Variante des Gesetzes verwendet,
welches der Quelle \cite{lorentz} entommen ist:
\begin{equation}
  \label{eq:lorentz}
  n\approx \sqrt{1+\frac{ 3Ap }{ \map{R}T } }.
\end{equation}
Die in Gleichung \eqref{eq:lorentz} auftretenden Größen sind,
zum einen die Gaskonstante $\map{R}$, die Temperatur $T$ und der
molare Brechungsindex $A$.
Die Abhängigkeit wird in einem Interferometer verwendet,um zwei kohärente Strahlen interfiereren zu lassen.
Durch eine Beobachtung bzw. Zählung der Interferenzmaxima $M$ bei einer gegebenen Druckdifferenz ist es möglich den Brechungsindex zu bestimmen.
Hierbei gilt für die Anzahl der Maxima bei einem gegbene Druck der folgende Zusammenhang:
\begin{equation}
  \label{eq:anzahl_maxima_gas}
  M=2\frac{\Delta\phi}{2\pi}= 2\frac{ n_B-n_A }{ \lambda\ua{vac} }L.
\end{equation}
Der Faktor $2$ in Formel \eqref{eq:anzahl_maxima_gas} ist damit zu begründen, dass der Laserstrahl, zweimal die Gaszelle durchquert.

\subsection{Brechungsindex von Festkörpern}
Im Gegensatz zu Gasen ist der Brechungsindex eines Festkörpers nicht Druckabhängig.
Deshalb muss hier eine andere Methdodik zur Ermittlung des Brechungsindexes verwendet werden.
Propagiert ein Lichtstrahl, wie in Abbildung \ref{fig:slag} gezeigt, durch einen Festkörper der Länge $T$,
so wird dieser nach dem Snelliussches Brechungsgesetz gebrochen.
\begin{figure}
\centering
\includegraphics[width=0.6\linewidth]{./content/images/slab.png}
\caption{Ein Lichtstahl propagiert durch einen Festkörper z.\,B. Glas der Länge $T$ \cite{anleitung64}.}
\label{fig:slag}
\end{figure}
Hierdurch entsteht ein Gangunterschied der sich in eine Phasendifferenz übersetzt.
Mathematisch bestimmt werden kann die Phasendifferenz über den folgenden Zusammenhang
\begin{equation}
  \label{eq:Phasendifferenz_festkörper}
  \Delta\phi=\frac{ 2\pi T }{ \lambda\ua{vac} }\left( \frac{ n-\cos(\theta-\theta') }{ \cos(\theta') } - n+1\right),
\end{equation}
wobei $n$ der Brechungsindex des Festkörpers repräsentiert.
\textbf{TAylorentwicklung}
Mit der Phasendifferenz ist es nun möglich die Anzahl an Interfernzmaxima $M$ anzugeben, die bei einer Änderung
des Festkörpers um den Winkel $\theta$ entstehen. Zu beachten ist, dass der Laserstrahl, in einem Interferometer, zweimal den Festkörper durchquert.
\begin{equation}
  \label{eq:Interferenzmaxima_solid_state}
  M=2\frac{\Delta\phi}{2\pi}=\textbf{HierNochTaylorentwicklungEinfügen}
\end{equation}
Wird die Gleichung \eqref{eq:Interferenzmaxima_solid_state} nach dem Brechungsindex $n$
umgestellt, ist es möglichen diesen aus $M$ zu bestimmen.
\begin{equation}
  \textbf{HiernochdieNeueFormeleinfügen}
\end{equation}

\subsection{Polarisation}
Bisher wurde lediglich der Wellencharakter von Licht betrachtet. Jedoch besitzt eine
elektromagnetische Welle einen weiteren Freiheitsgrad, die Polarisation.
Die Polarisation gibt die Richtung des elektrischenfeld Vektors an.
Eine Änderung der Polarisation kann mit Hilfe von Polarisationsfiltern bewirkt werden.
Der Filter absorbiert den Lichtanteil, welcher um $\frac{\pi}{2}$ zum eingestellt Filterwinkel
gedreht ist. Somit wird die Polarisation des transmitierte Lichtes auf die Filterachse projiziert (vgl. Abb. \ref{fig:pola}).
\begin{figure}
\centering
\includegraphics[width=0.5\linewidth]{./content/images/polarisationsfilter.png}
\caption{Schematische Darstellung der Polarisationsänderung durch ein Polarisationsfilter \cite{pola}.}
\label{fig:pola}
\end{figure}

Zusätzlich werden in dem Versuch \emph{polarizing beam-splitter cubes} (PBSCs) verwendet.
Hergestellt wird ein solche Würfel, wenn zwei $45-45-\SI{90}{\degree}$ Prismen
an der Hypotenuse mit einander verklebt werden. Wichtig ist das auf dieser, vor dem verkleben,
eine dielekrische Schicht aufgebracht wird. Ein PBSC teilt den auftrefenden Strahl
in zwei Strahlen auf, die orthogonal in Ausbreitungsrichtung und Polarisation
sind (vgl. Abb. \ref{fig:pbsc}).
\begin{figure}
\centering
\includegraphics[width=0.6\linewidth]{./content/images/pbsc.png}
\caption{Schematische Darstellung eines PCBS \cite{anleitung64}.}
\label{fig:pbsc}
\end{figure}
