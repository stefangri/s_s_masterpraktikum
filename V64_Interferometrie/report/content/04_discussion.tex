\section{Diskussion}
Der maximal gemessene Kontrast des Interferometers von ca. $\SI{96}{\percent}$ spricht für 
eine gelungene Justage des Aufbaus. Der Verlauf des Kontrasts als Funktion des Polarisationswinkel 
(Abbildung\ref{}) zeigt den erwarteten Verlauf. Die Fitparamter in Model~\eqref{} weichen mit 
$A = $ in plausibler Weise von den theoretisch bzw. ideal erwarteten Werten ($A = 1$). Insbesondere 
der nicht verschwindende Wert für $\vartheta_0$ ist darauf zurück zu führen, dass der \ce{HeNe}-Laser 
nicht exakt um $\SI{45}{\degree}$ gegen die Vertikale angestellt ist.

Für den Brechungsindex von Glas findet man z.B. in \cite{} einen Wert von $\SI{1.5}{}$. Der hier bestimmte 
Wert weicht mit $\input{../analysis/results/result_n_glas.tex}$ nach unten hiervon ab. Gleiches gilt 
für den gefunden Brechungsindex von Luft, $\input{../analysis/results/result_n.tex}$. Bei einer Temperatur von $\SI{24}{\celsius}$, die 
während des Experiments gemessen wurde, ist nach \cite{} ein Brechungsindex von zu $\SI{}{}$ erwarten. Sowohl die 
Funktion~\eqref{}, als auch die Funktion~\eqref{} verknüpfen eine zu kleine Zählrate monoton mit einem zu kleinen 
Brechungsindex. Daher kann hier als Grund für die Abweichungen der bestimmten Brechungsindices eine 
fehlerhafte Zählelektronik zurück geführt werden, die offenbar systematisch zu wenige $2\pi$-Phasenverschiebungen aufnimmt. 
Eine Ersetzung der Zählelektronik durch eine ausschließliche Vermessung mit dem Oszilloskop ist zu empfehlen.
