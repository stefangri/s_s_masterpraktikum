\newpage
\section{Diskussion}
Der maximal gemessene Kontrast des Interferometers von ca. $\SI{96}{\percent}$ spricht für 
eine gelungene Justage des Aufbaus. Der Verlauf des Kontrasts als Funktion des Polarisationswinkel 
(Abbildung~\ref{fig: kontrast_fit}) zeigt den erwarteten Verlauf. Die Fitparamter in Model~\eqref{eq: model_contrast} weichen mit 
$A = \input{../analysis/results/result_A.tex}$, $\vartheta_0 = \input{../analysis/results/result_vartheta_0.tex}$ 
und $C = \input{../analysis/results/result_C.tex}$ in plausibler Weise von den theoretisch bzw. ideal erwarteten Werten 
($A = 1$, $\vartheta_0 = C = 0$). Insbesondere 
der nicht verschwindende Wert für $\vartheta_0$ ist darauf zurück zu führen, dass der \ce{HeNe}-Laser 
nicht exakt um $\SI{45}{\degree}$ gegen die Vertikale angestellt ist.

Für den Brechungsindex von Glas findet man z.B. in \cite{rii} einen Wert von etwa $\SI{1.5}{}$. Der hier bestimmte 
Wert von $\input{../analysis/results/result_n_glas.tex}$ scheint somit realistisch. Gleiches gilt 
für den gefunden Brechungsindex von Luft, $\input{../analysis/results/result_n.tex}$. Bei einer Temperatur von $\SI{24}{\celsius}$, die 
während des Experiments gemessen wurde, ist nach \cite{refr_index_air} ein Brechungsindex von ca. $\SI{1.000265}{}$ zu erwarten. 
Die Messung liefert also eine hervorragende Übereinstimmung und das hier verwendete Interferometer ist als 
sehr präzise einzustufen.
