\section{Procedure and results}
Besides the experiemntell procedure, this section presents as main results the
absorption spectrum of Rubidium. Therefore, the setup shown in figure \ref{fig: setup}
is used. Before an examination of the spectrum is possible, the laser system has to be
adjust correctly.

To operate the setup properly, the Rubidium gas cell must be heated
continuously to $\SI{50}{\degreeCelsius}$. During the heating process, a infrared display
card is inserted in the path of the later laser beam. The diode current is increased slowly until a
light spot is visible on the display card. Therefore, a camera is used.
Depending on the current are
two emission modes of the laser observerable as presented in figure \ref{fig: emission_modes}.
\begin{figure}
  \centering
      \begin{subfigure}{0.48\textwidth}
          \centering
          \includegraphics[width = \textwidth]{./content/images/diodelaser_not_lasering.jpg}
          \caption{Not lasering.}
          \label{fig:not_lasering}
      \end{subfigure}
      \begin{subfigure}{0.48\textwidth}
          \centering
          \includegraphics[width = \textwidth]{./content/images/diodelaser_lasering.jpg}
          \caption{Lasering.}
          \label{fig:lasering}
      \end{subfigure}
  \caption{Observerd emission modes of the laser. The diode laser behaves like a regular
  LED under a certain current threshold, \ref{fig:not_lasering}. After reaching the
  current threshold the diode laser emits coherent light \ref{fig:lasering}. }
\label{fig: emission_modes}
\end{figure}
If the current reaches a certain threshold, the diode starts to emit coherent
light, and can be used as laser. To operate the diode laser efficently,
the current is decreased slightly below the threshold. The light spot on the
diplay card is now equal to image \ref{fig:not_lasering}. By moving the knobs
on the diffraction grating carefully, the diode starts to emit coherent light
as shown in image \ref{fig:lasering}. The knobs slightly changes the length of
the external cavity and how the light reflected from the grating hits the diode.
The process is repeated as long as it is possible to get coherent light. The minimal
current is determine with a voltage measurement. With the resistance
of the current source ($\SI{100}{\ohm}$) and Ohm's law the current can be calculated
to $I\ua{min}=\SI{0.33}{\ampere}$. As next step, the dipslay card has to be removed
and a Rubidium gas cell is mounted into the laser path. As shown in figure \ref{fig: positioning_cell_camera},
the camera film the center of the cell.
\begin{figure}
  \centering
  \includegraphics[width = 0.8\textwidth]{./content/images/cell_camera_heater.png}
  \caption{Positioning of cell and camera.}
  \label{fig: positioning_cell_camera}
\end{figure}
To decrease the intensty of the laser beam before entering the cell, ND filters can be used.
The diode current is increased slowly until the camera observe the fluorescence
of Rubidium. Rubidium is only stimulated within the laser beam. Hence,
the characteristic fluorescence shape should be a line respective a cylinder
as demonstrated in figure \ref{fig: fluro_rubidium}.
\begin{figure}
  \centering
  \includegraphics[width = 0.5\textwidth]{./content/images/emission_of_Ga.jpg}
  \caption{Fluorescence of Rubidium (line).}
  \label{fig: fluro_rubidium}
\end{figure}
