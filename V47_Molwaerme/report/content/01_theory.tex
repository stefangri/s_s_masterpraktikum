\setcounter{page}{1}
\section*{Zielsetzung}
In dem Versuch V47 wird die Molwärme von Kupfer $\ce{Cu}$ bestimt werden.
Molwärme wird hierbei als Synonym für die \emph{molare Wärmekapazität} verwendet.
Diese gibt die Menge an Wärme an, die benötigt ist um $\SI{1}{\mol}$ eines Stoffes
bzw. Elements um einen $\SI{1}{\kelvin}$ zu erwärmen.

\section{Theorie}
Im Folgenden werden drei Methodiken zur Bestimmung der Molwärme erläutert.
Zunächst wird die Molwärme klassisch hergeleitet. Anschließend werden zwei
quantenmechanische Modelle von \emph{Einstein} und \emph{Debye}
vorgestellt. Hierfür werden die Quellen \cite{} und \cite{}[S.215] verwendet.

Weiterhin kann die Molwärme bei unterschiedlichen Bedingungen betrachtet werden.
Wird der äußere Druck $p$ festgehalten, wird von der Wärmekapazität bei konstantem
Druck gesprochen. Aus der Thermodynamik folgt für diesen Fall der Zusammenhang:
\begin{equation}
  \label{eq:C_P}
  C\ua{p} = \left.\frac{\partial Q}{\partial T}\right|_{\map{p}}.
\end{equation}
Anstatt des Druckes $p$ kann auch das Volumen des Festkörpers $V$ festgehalten
werden. Dies definiert die Wärmekapazität bei konstantem Volumen:
\begin{equation}
  \label{eq:C_V}
  C\ua{V} = \left.\frac{\partial Q}{\partial T}\right|_{\map{V}} = \left. \frac{\partial U}{\partial T}\right|_{\map{V}}.
\end{equation}
Auf Grund der Tatsache das sich $C\ua{V}$ aus der inneren Energie $U$ ableiten
lässt, ist die Beschreibung über ein theoretisches Modell leichter zu realisieren.
Jedoch kann in einem Experiment kaum sichergestellt werden, dass das Volumen einer
Probe konstant bleibt. Aus diesem Grund wird meist $C_p$ gemssen und mit Hilfe
der folgenden Formel
\begin{equation}
  \label{eq:Umrechnung_CP_CV}
  C\ua{p} - C\ua{V} = 9\alpha^2 \kappa V_0 T
\end{equation}
umgerechnet werden. Die in der Gleichun \eqref{eq:Umrechnung_CP_CV} auftretenden
Größen sind - der lineare Ausdehungskoeffizient $\alpha$, das Kompressionsmodul
$\kappa$ und das Molvolumen $V_0$.
\subsection{Klassische Betrachtung}
