\section{Auswertung}



\subsection{Vorbereitende Berechnungen}
Für spätere Berechnungen ist zunächst der Zusammenhang zwischen dem Widerstand der \ce{Pt}-Elemente und der Temperatur 
zu herzustellen. Hierzu sind im Skript~\cite{anleitung47} die Daten aus Tabelle~\ref{} gegeben, die durch die quadratische 
Funktion
\begin{equation}
    T(R) = \SI{0.00134}{\kelvin \per \ohm\squared} R^2 + \SI{2.296}{\kelvin \per \ohm} R + \SI{30.13}{\kelvin}
    \label{eq: T_R}
\end{equation}
beschrieben werden. Zudem ist für die Umrechnung zwischen $C_p$ und $C_V$ mit Gleichung~\eqref{} der Temperaturverlauf 
$\alpha(T)$ des Extinktionskoeffizienten von Interesse. Auch hierfür sind vorgegebene Datenpunkte aus~\cite{} entnommen 
worden und in Tabelle~\ref{} eingetragen. Die Daten werden linear interpoliert und somit später Werte für $\alpha$ für 
beliebige Temperaturen zwischen $\SI{80}{\kelvin}$ und $\SI{300}{\kelvin}$ bestimmt. Von einer Betrachtung von Fehlern 
des Modells wird wie in~\eqref{eq: T_R} abgesehen. Einer graphische Darstellung der Messpunkte und angenommenen 
Modelle befindet sich in Abbildung~\ref{fig: alpha_T_R}.
\begin{figure}
\centering
\includegraphics[scale = 1]{../analysis/results/alpha_T_R.pdf}
\caption{(a) Darstellung der Messwerte aus Tabelle~\ref{tab: } zur Beschreibung des Zusammenhangs zwischen Temperatur 
        $T$ und Widerstand $R$.
        (b) Darstellung der Messwerte aus Tabelle~\ref{tab: } zur Bestimmung des Zusammenhangs zwischen Temperatur und Ausdehnungskoeffizient $\alpha$.}
\label{fig: alpha_T_R}
\end{figure}


\subsection{Bestimmung der Debye-Temperatur von Kupfer}


Als Mittelwert ergibt sich: 
\begin{equation}
        \Theta\ua{D, exp} = \input{../analysis/results/result_T_debye_exp.tex}
\end{equation}