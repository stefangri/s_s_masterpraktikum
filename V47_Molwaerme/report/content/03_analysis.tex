\section{Auswertung}
Nachfolgend werden die durchgeführten Berechnung zur Bestimmung der Debye-Temperatur von Kupfer 
vorgestellt. Allgemein werden Fehler gemäß der Gaußschen Fehlerfortpflanzung berechnet, eine mögliche 
Korrelation der gemessenen Größen wird nicht beachtet.

\subsection{Vorbereitende Berechnungen}
Für spätere Berechnungen ist zunächst der Zusammenhang zwischen dem Widerstand der \ce{Pt}-Elemente und der Temperatur 
herzustellen. Hierzu sind im Skript~\cite{anleitung47} die Daten aus Tabelle~\ref{tab: T_R} gegeben, die durch die quadratische 
Funktion
\begin{equation}
    T(R) = \SI{0.00134}{\kelvin \per \ohm\squared} R^2 + \SI{2.296}{\kelvin \per \ohm} R + \SI{30.13}{\kelvin}
    \label{eq: T_R}
\end{equation}
beschrieben werden. Zudem ist für die Umrechnung zwischen $C_p$ und $C_V$ mit Gleichung~\eqref{} der Temperaturverlauf 
$\alpha(T)$ des Ausdehnungskoeffizienten von Interesse. Auch hierfür sind vorgegebene Datenpunkte aus~\cite{anleitung47} entnommen 
worden und in Tabelle~\ref{tab: alpha} eingetragen. Die Daten werden linear interpoliert und somit Werte von $\alpha$ für 
beliebige Temperaturen zwischen $\SI{80}{\kelvin}$ und $\SI{300}{\kelvin}$ bestimmt. Von einer Betrachtung der Ungenauigkeit 
dieses Modells wird wie in~\eqref{eq: T_R} abgesehen. Eine graphische Darstellung der Messpunkte und angenommenen 
Modelle befindet sich in Abbildung~\ref{fig: alpha_T_R}.
\begin{figure}
\centering
\includegraphics[scale = 1]{../analysis/results/alpha_T_R.pdf}
\caption{(a) Darstellung der Messwerte aus Tabelle~\ref{tab: T_R} zur Beschreibung des Zusammenhangs zwischen Temperatur 
        $T$ und Widerstand $R$.
        (b) Darstellung der Messwerte aus Tabelle~\ref{tab: } zur Bestimmung des Zusammenhangs zwischen Temperatur und
         Ausdehnungskoeffizient $\alpha$.}
\label{fig: alpha_T_R}
\end{figure}

\begin{table}
	\centering
	\caption{Aus der Anleitung~\cite{anleitungV47} entnommene Wertepaare für $T$ und $\alpha$.}
	\label{tab: alpha}
	\begin{tabular}{
		S[table-format=3.0]
		S[table-format=2.1]
		S[table-format=3.0]
		S[table-format=2.1]
		}
	\toprule
		{$T$\;/\;\si{\kelvin}} &
		{$\alpha$\;/\;\si{10^{-6}\per \kelvin}} &
		{$T$\;/\;\si{\kelvin}} &
		{$\alpha$\;/\;\si{10^{-6}\per \kelvin}} \\
	\midrule
		 70 &  7.0 &  190 &  14.8 \\
		 80 &  8.5 &  200 &  14.9 \\
		 90 &  9.8 &  210 &  15.2 \\
		 100 &  10.7 &  220 &  15.4 \\
		 110 &  11.5 &  230 &  15.6 \\
		 120 &  12.1 &  240 &  15.8 \\
		 130 &  12.7 &  250 &  15.9 \\
		 140 &  13.2 &  260 &  16.1 \\
		 150 &  13.6 &  270 &  16.2 \\
		 160 &  13.9 &  280 &  16.4 \\
		 170 &  14.2 &  290 &  16.5 \\
		 180 &  14.5 &  300 &  16.6 \\
	\bottomrule
	\end{tabular}
\end{table}
\begin{table}
	\centering
	\caption{Aus der Anleitung~\cite{anleitung47} entnommene Wertepaare für $R$ und $T$.}
	\label{tab: T_R}
	\begin{tabular}{
		S[table-format=3.2]
		S[table-format=3.2]
		S[table-format=3.2]
		S[table-format=3.2]
		}
	\toprule
		{$R$\;/\;\si{\ohm}} &
		{$T$\;/\;\si{\kelvin}} &
		{$R$\;/\;\si{\ohm}} &
		{$T$\;/\;\si{\kelvin}} \\
	\midrule
		 18.44 &  73.15 &  72.29 &  203.15 \\
		 22.71 &  83.15 &  76.28 &  213.15 \\
		 27.03 &  93.15 &  80.25 &  223.15 \\
		 31.28 &  103.15 &  84.21 &  233.15 \\
		 35.48 &  113.15 &  88.17 &  243.15 \\
		 39.65 &  123.15 &  92.13 &  253.15 \\
		 43.80 &  133.15 &  96.07 &  263.15 \\
		 47.93 &  143.15 &  100.00 &  273.15 \\
		 52.04 &  153.15 &  103.90 &  283.15 \\
		 56.13 &  163.15 &  107.79 &  293.15 \\
		 60.20 &  173.15 &  111.67 &  303.15 \\
		 64.25 &  183.15 &  115.54 &  313.15 \\
		 68.28 &  193.15 &  &  \\
	\bottomrule
	\end{tabular}
\end{table}


\subsection{Bestimmung der Debye-Temperatur von Kupfer}
Die gewonnen Messdaten für den Widerstand der Probe und des Zylinders, sowie der Spannung und des Stroms
als Funktion der Zeit sind in Tabelle~\ref{tab: data} aufgeführt. Für alle Daten wird eine $1$ auf der letzten
Nachkommastelle der Messwerte als Fehler angenommen. 
Mit dem dem Modell~\eqref{eq: T_R} ergeben sich aus den Widerständen die Temperaturen von Probe und Zylinder.
Die Temperaturverläufe sind zudem in Abbildung~\ref{fig: temp}(a) gezeigt. Wie im vorherigen Abschnitt beschrieben, ist es 
wichtig, dass die Temperaturen von Zylinder und Probe stets gleich sind. Da in Abbildung~\ref{fig: temp}(a) Abweichungen nicht 
deutlich zu erkennen sind, ist in Abbildung~\ref{fig: temp}(b) ebenfalls die Differnez der beiden Temperaturen 
aufgetragen. Hier wird ersichtlich, dass die Temperaturabweichungen im Mittel sehr klein gehalten werden konnten, es
jedoch vereinzelt zu Temperaturdifferenzen bis zu $\sim\!\SI{4}{\kelvin}$ gekommen ist.
\begin{figure}
\centering
\includegraphics[scale = 1]{../analysis/results/temperaturen.pdf}
\caption{Zeitlicher Verlauf der Temperaturen von Zylinder und Probe (a), sowie deren Differenz (b). In Rot sind 
        zwei Datenpunkte eingezeichnet, die später diskutiert werden.}
\label{fig: temp}
\end{figure}

\begin{table}
	\centering
	\caption{Aufgenommene Messdaten zur Bestimmung der Debye Temperatur von Kupfer. Teit $t$, anliegende Spannung $U$ an der Probe, Strom $I$ durch die Probe, Widerstand $R_{\text{zyl}}$ des Zylinder-Pt-Elements und Widerstand $R_{\text{probe}}$ des Proben-Pt-Elements. Aus den Widerständen werden die Temperaturen $T_{\text{zyl}}$ und $T_{\text{probe}}$ berechnet.}
	\label{tab: data}
\resizebox{\textwidth}{!}{
	\begin{tabular}{
		S[table-format=4.0] @{${}\pm{}$} S[table-format=1.0]
		S[table-format=2.2] @{${}\pm{}$} S[table-format=1.2]
		S[table-format=3.1] @{${}\pm{}$} S[table-format=1.1]
		S[table-format=1.4] @{${}\pm{}$} S[table-format=1.4]
		S[table-format=3.2] @{${}\pm{}$} S[table-format=1.2]
		S[table-format=1.4] @{${}\pm{}$} S[table-format=1.4]
		S[table-format=3.2] @{${}\pm{}$} S[table-format=1.2]
		}
	\toprule
		\multicolumn{2}{c}{$t$\;/\;\si{\second}} &
		\multicolumn{2}{c}{$U$\;/\;\si{\volt}} &
		\multicolumn{2}{c}{$I$\;/\;\si{\milli\ampere}} &
		\multicolumn{2}{c}{$R_{\text{zyl}}$\;/\;\si{\kilo\ohm}} &
		\multicolumn{2}{c}{$T_{\text{zyl}}$\;/\;\si{\kelvin}} &
		\multicolumn{2}{c}{$R_{\text{prob}}$\;/\;\si{\kilo\ohm}} &
		\multicolumn{2}{c}{$T_{\text{prob}}$\;/\;\si{\kelvin}} \\
	\midrule
		   0 &  1 &  15.75 &  0.01 &  150.4 &  0.1 &  0.0217 &  0.0001 &  80.58 &  0.24 &  0.0210 &  0.0001 &  78.94 &  0.24 \\
		 300 &  1 &  15.95 &  0.01 &  152.3 &  0.1 &  0.0230 &  0.0001 &  83.65 &  0.24 &  0.0249 &  0.0001 &  88.13 &  0.24 \\
		 600 &  1 &  11.95 &  0.01 &  114.0 &  0.1 &  0.0260 &  0.0001 &  90.73 &  0.24 &  0.0273 &  0.0001 &  93.81 &  0.24 \\
		 900 &  1 &  15.80 &  0.01 &  150.6 &  0.1 &  0.0293 &  0.0001 &  98.55 &  0.24 &  0.0296 &  0.0001 &  99.27 &  0.24 \\
		 1200 &  1 &  15.86 &  0.01 &  151.0 &  0.1 &  0.0322 &  0.0001 &  105.45 &  0.24 &  0.0328 &  0.0001 &  106.88 &  0.24 \\
		 1500 &  1 &  16.38 &  0.01 &  155.8 &  0.1 &  0.0350 &  0.0001 &  112.13 &  0.24 &  0.0359 &  0.0001 &  114.28 &  0.24 \\
		 1800 &  1 &  16.41 &  0.01 &  155.9 &  0.1 &  0.0383 &  0.0001 &  120.03 &  0.24 &  0.0390 &  0.0001 &  121.71 &  0.24 \\
		 2100 &  1 &  16.43 &  0.01 &  156.0 &  0.1 &  0.0417 &  0.0001 &  128.20 &  0.24 &  0.0419 &  0.0001 &  128.68 &  0.24 \\
		 2400 &  1 &  16.45 &  0.01 &  156.1 &  0.1 &  0.0454 &  0.0001 &  137.13 &  0.24 &  0.0449 &  0.0001 &  135.92 &  0.24 \\
		 2700 &  1 &  16.47 &  0.01 &  156.2 &  0.1 &  0.0486 &  0.0001 &  144.88 &  0.24 &  0.0479 &  0.0001 &  143.18 &  0.24 \\
		 3000 &  1 &  16.48 &  0.01 &  156.3 &  0.1 &  0.0510 &  0.0001 &  150.71 &  0.25 &  0.0510 &  0.0001 &  150.71 &  0.25 \\
		 3300 &  1 &  16.49 &  0.01 &  156.3 &  0.1 &  0.0533 &  0.0001 &  156.31 &  0.25 &  0.0538 &  0.0001 &  157.53 &  0.25 \\
		 3600 &  1 &  16.50 &  0.01 &  156.4 &  0.1 &  0.0553 &  0.0001 &  161.20 &  0.25 &  0.0564 &  0.0001 &  163.89 &  0.25 \\
		 3900 &  1 &  16.51 &  0.01 &  156.4 &  0.1 &  0.0587 &  0.0001 &  169.52 &  0.25 &  0.0590 &  0.0001 &  170.26 &  0.25 \\
		 4200 &  1 &  16.51 &  0.01 &  156.4 &  0.1 &  0.0623 &  0.0001 &  178.37 &  0.25 &  0.0617 &  0.0001 &  176.90 &  0.25 \\
		 4500 &  1 &  16.52 &  0.01 &  156.4 &  0.1 &  0.0657 &  0.0001 &  186.76 &  0.25 &  0.0644 &  0.0001 &  183.55 &  0.25 \\
		 4800 &  1 &  17.05 &  0.01 &  161.2 &  0.1 &  0.0690 &  0.0001 &  194.94 &  0.25 &  0.0675 &  0.0001 &  191.22 &  0.25 \\
		 5100 &  1 &  17.60 &  0.01 &  166.6 &  0.1 &  0.0712 &  0.0001 &  200.40 &  0.25 &  0.0705 &  0.0001 &  198.66 &  0.25 \\
		 5400 &  1 &  17.61 &  0.01 &  166.7 &  0.1 &  0.0735 &  0.0001 &  206.13 &  0.25 &  0.0734 &  0.0001 &  205.88 &  0.25 \\
		 5700 &  1 &  17.62 &  0.01 &  166.8 &  0.1 &  0.0760 &  0.0001 &  212.37 &  0.25 &  0.0762 &  0.0001 &  212.87 &  0.25 \\
		 6000 &  1 &  17.63 &  0.01 &  166.9 &  0.1 &  0.0790 &  0.0001 &  219.88 &  0.25 &  0.0790 &  0.0001 &  219.88 &  0.25 \\
		 6300 &  1 &  17.63 &  0.01 &  166.9 &  0.1 &  0.0825 &  0.0001 &  228.67 &  0.25 &  0.0818 &  0.0001 &  226.91 &  0.25 \\
		 6600 &  1 &  17.63 &  0.01 &  167.0 &  0.1 &  0.0846 &  0.0001 &  233.97 &  0.25 &  0.0845 &  0.0001 &  233.71 &  0.25 \\
		 6900 &  1 &  17.63 &  0.01 &  167.0 &  0.1 &  0.0860 &  0.0001 &  237.50 &  0.25 &  0.0871 &  0.0001 &  240.28 &  0.25 \\
		 7200 &  1 &  17.63 &  0.01 &  167.0 &  0.1 &  0.0894 &  0.0001 &  246.11 &  0.26 &  0.0896 &  0.0001 &  246.62 &  0.26 \\
		 7500 &  1 &  18.06 &  0.01 &  171.2 &  0.1 &  0.0925 &  0.0001 &  253.98 &  0.26 &  0.0924 &  0.0001 &  253.73 &  0.26 \\
		 7800 &  1 &  18.07 &  0.01 &  171.3 &  0.1 &  0.0956 &  0.0001 &  261.88 &  0.26 &  0.0953 &  0.0001 &  261.12 &  0.26 \\
		 8100 &  1 &  18.08 &  0.01 &  171.4 &  0.1 &  0.0983 &  0.0001 &  268.78 &  0.26 &  0.0981 &  0.0001 &  268.27 &  0.26 \\
		 8400 &  1 &  18.65 &  0.01 &  176.8 &  0.1 &  0.1012 &  0.0001 &  276.22 &  0.26 &  0.1011 &  0.0001 &  275.96 &  0.26 \\
		 8700 &  1 &  18.66 &  0.01 &  177.0 &  0.1 &  0.1039 &  0.0001 &  283.16 &  0.26 &  0.1041 &  0.0001 &  283.67 &  0.26 \\
		 9000 &  1 &  18.67 &  0.01 &  177.1 &  0.1 &  0.1073 &  0.0001 &  291.93 &  0.26 &  0.1070 &  0.0001 &  291.15 &  0.26 \\
		 9300 &  1 &  18.67 &  0.01 &  177.2 &  0.1 &  0.1098 &  0.0001 &  298.40 &  0.26 &  0.1099 &  0.0001 &  298.66 &  0.26 \\
		 9600 &  1 &  18.67 &  0.01 &  177.3 &  0.1 &  0.1122 &  0.0001 &  304.62 &  0.27 &  0.1127 &  0.0001 &  305.92 &  0.27 \\
	\bottomrule
	\end{tabular}}
\end{table}

Aus den Daten kann nun zunächst die Wärmekapazität unter konstantem Druck $C_p$ bestimmt werden. Hierzu werden 
die Temperaturdifferenzen und Zeitdifferenzen aus den Daten in Tabelle~\ref{tab: data} gebildet und zusammen mit den 
zugehörigen Spannungen und Strömen in Formel~\eqref{} eingesetzt. Für die Molare Masse von Kupfer 
wird $\SI{63.5}{\gram / \mol}$ \cite{molarmass} angenommen, die Probe hat eine Masse von $\SI{342}{\gram}$ \cite{anleitung47}. 
Zur Berechnung der Wärmekapazität
bei konstantem Volumen wird Gleichung~\eqref{} verwendet. Die hierhin auftretende Temperatur wird als Mittel 
der beiden Temperaturen, aus denen die Temperaturdifferenz für Gleichung~\eqref{} bestimmt wurde, angenommen. 
Für die Materialgrößen werden die folgenden Werte verwendet: $\kappa = \SI{14e10}{\pascal}$ \cite{gross}, $V_0 = \SI{7.0922e-6}{\meter^3 / \mol}$ \cite{volume}.
Der letzte Wert übersteigt den Wert $\SI{300}{\kelvin}$ und ist damit nicht mehr mit dem Modell für $\alpha(T)$
zu beschreiben. Er wird daher nicht berücksichtigt.

\begin{table}
	\centering
	\caption{Ergebnisse für die spezifische Wärmekapazität bei konstantem Druck und konstantem Volumen, sowie der Debye-Temperatur.}
	\label{tab: results}
\resizebox{\textwidth}{!}{
	\begin{tabular}{
		S[table-format=3.1] @{${}\pm{}$} S[table-format=1.1]
		S[table-format=1.2] @{${}\pm{}$} S[table-format=1.2]
		S[table-format=2.1] @{${}\pm{}$} S[table-format=1.1]
		S[table-format=3.2] @{${}\pm{}$} S[table-format=1.2]
		S[table-format=2.2]
		S[table-format=2.1] @{${}\pm{}$} S[table-format=1.1]
		S[table-format=1.1] @{${}\pm{}$} S[table-format=1.1]
		S[table-format=3.0] @{${}\pm{}$} S[table-format=3.0]
		}
	\toprule
		\multicolumn{2}{c}{$\Delta t$\;/\;\si{\second}} &
		\multicolumn{2}{c}{$\Delta T$\;/\;\si{\kelvin}} &
		\multicolumn{2}{c}{$C_p$\;/\;\si{\joule\per\kelvin\per\mol}} &
		\multicolumn{2}{c}{$\overline{T}$\;/\;\si{\kelvin}} &
		{$\alpha(\overline{T})$\;/\;\si{10^{-6}\per\kelvin}} &
		\multicolumn{2}{c}{$C_V$\;/\;\si{\joule\per\kelvin\per\mol}} &
		\multicolumn{2}{c}{$\frac{\Theta}{\overline{T}}$} &
		\multicolumn{2}{c}{$\Theta_D$\;/\;\si{\kelvin}} \\
	\midrule
		 300.0 &  1.4 &  9.19 &  0.33 &  14.4 &  0.5 &  83.53 &  0.18 &  8.94 &  14.3 &  0.5 &  3.5 &  0.2 &  292 &  17 \\
		 300.0 &  1.4 &  5.68 &  0.33 &  23.8 &  1.4 &  90.97 &  0.17 &  9.84 &  23.8 &  1.4 &  1.0 &  1.0 &  91 &  91 \\
		 300.0 &  1.4 &  5.46 &  0.34 &  13.9 &  0.9 &  96.54 &  0.17 &  10.37 &  13.8 &  0.9 &  3.7 &  0.2 &  357 &  19 \\
		 300.0 &  1.4 &  7.61 &  0.34 &  17.4 &  0.8 &  103.07 &  0.17 &  10.95 &  17.3 &  0.8 &  2.8 &  0.2 &  289 &  21 \\
		 300.0 &  1.4 &  7.40 &  0.34 &  18.0 &  0.8 &  110.58 &  0.17 &  11.53 &  17.9 &  0.8 &  2.7 &  0.2 &  299 &  22 \\
		 300.0 &  1.4 &  7.43 &  0.34 &  19.1 &  0.9 &  118.00 &  0.17 &  11.98 &  19.0 &  0.9 &  2.4 &  0.2 &  283 &  24 \\
		 300.0 &  1.4 &  6.97 &  0.34 &  20.5 &  1.0 &  125.20 &  0.17 &  12.39 &  20.3 &  1.0 &  2.1 &  0.3 &  263 &  38 \\
		 300.0 &  1.4 &  7.24 &  0.34 &  19.7 &  0.9 &  132.30 &  0.17 &  12.77 &  19.5 &  0.9 &  2.3 &  0.3 &  304 &  40 \\
		 300.0 &  1.4 &  7.26 &  0.34 &  19.7 &  0.9 &  139.55 &  0.17 &  13.13 &  19.5 &  0.9 &  2.3 &  0.3 &  321 &  42 \\
		 300.0 &  1.4 &  7.53 &  0.34 &  19.0 &  0.9 &  146.95 &  0.17 &  13.46 &  18.8 &  0.9 &  2.4 &  0.3 &  353 &  44 \\
		 300.0 &  1.4 &  6.82 &  0.34 &  21.0 &  1.1 &  154.12 &  0.18 &  13.72 &  20.8 &  1.1 &  1.9 &  0.3 &  293 &  46 \\
		 300.0 &  1.4 &  6.35 &  0.35 &  22.6 &  1.2 &  160.71 &  0.18 &  13.92 &  22.3 &  1.2 &  1.5 &  0.4 &  241 &  64 \\
		 300.0 &  1.4 &  6.37 &  0.35 &  22.6 &  1.2 &  167.07 &  0.18 &  14.15 &  22.3 &  1.2 &  1.5 &  0.4 &  251 &  67 \\
		 300.0 &  1.4 &  6.64 &  0.35 &  21.7 &  1.1 &  173.58 &  0.18 &  14.34 &  21.4 &  1.1 &  1.8 &  0.4 &  312 &  69 \\
		 300.0 &  1.4 &  6.66 &  0.35 &  21.6 &  1.1 &  180.22 &  0.18 &  14.51 &  21.3 &  1.1 &  1.8 &  0.3 &  324 &  54 \\
		 300.0 &  1.4 &  7.67 &  0.35 &  18.8 &  0.9 &  187.38 &  0.18 &  14.68 &  18.4 &  0.9 &  2.5 &  0.3 &  468 &  56 \\
		 300.0 &  1.4 &  7.44 &  0.35 &  20.6 &  1.0 &  194.94 &  0.18 &  14.85 &  20.2 &  1.0 &  2.1 &  0.3 &  409 &  58 \\
		 300.0 &  1.4 &  7.22 &  0.35 &  22.6 &  1.1 &  202.27 &  0.18 &  15.01 &  22.2 &  1.1 &  1.5 &  0.4 &  303 &  81 \\
		 300.0 &  1.4 &  6.99 &  0.35 &  23.4 &  1.2 &  209.37 &  0.18 &  15.18 &  23.0 &  1.2 &  1.3 &  0.5 &  272 &  105 \\
		 300.0 &  1.4 &  7.01 &  0.35 &  23.4 &  1.2 &  216.38 &  0.18 &  15.33 &  22.9 &  1.2 &  1.3 &  0.5 &  281 &  108 \\
		 300.0 &  1.4 &  7.03 &  0.36 &  23.3 &  1.2 &  223.40 &  0.18 &  15.47 &  22.8 &  1.2 &  1.3 &  0.4 &  290 &  89 \\
		 300.0 &  1.4 &  6.80 &  0.36 &  24.1 &  1.3 &  230.31 &  0.18 &  15.60 &  23.6 &  1.3 &  1.1 &  0.9 &  253 &  207 \\
		 300.0 &  1.4 &  6.57 &  0.36 &  25.0 &  1.4 &  237.00 &  0.18 &  15.70 &  24.5 &  1.4 &  0.6 &  0.6 &  142 &  142 \\
		 300.0 &  1.4 &  6.33 &  0.36 &  25.9 &  1.5 &  243.45 &  0.18 &  15.80 &  25.4 &  1.5 &  0.0 &  0.9 &   0 &  219 \\
		 300.0 &  1.4 &  7.11 &  0.36 &  23.1 &  1.2 &  250.17 &  0.18 &  15.90 &  22.5 &  1.2 &  1.4 &  0.4 &  350 &  100 \\
		 300.0 &  1.4 &  7.39 &  0.36 &  23.3 &  1.1 &  257.42 &  0.18 &  16.05 &  22.7 &  1.1 &  1.4 &  0.5 &  360 &  129 \\
		 300.0 &  1.4 &  7.16 &  0.36 &  24.1 &  1.2 &  264.69 &  0.18 &  16.17 &  23.5 &  1.2 &  1.1 &  0.7 &  291 &  185 \\
		 300.0 &  1.4 &  7.69 &  0.36 &  22.5 &  1.1 &  272.12 &  0.18 &  16.27 &  21.8 &  1.1 &  1.7 &  0.4 &  463 &  109 \\
		 300.0 &  1.4 &  7.71 &  0.36 &  23.8 &  1.1 &  279.82 &  0.19 &  16.35 &  23.2 &  1.1 &  1.2 &  0.5 &  336 &  140 \\
		 300.0 &  1.4 &  7.48 &  0.36 &  24.6 &  1.2 &  287.41 &  0.19 &  16.46 &  23.9 &  1.2 &  0.9 &  0.9 &  259 &  259 \\
		 300.0 &  1.4 &  7.50 &  0.37 &  24.6 &  1.2 &  294.91 &  0.19 &  16.57 &  23.8 &  1.2 &  1.0 &  1.0 &  295 &  295 \\
	\bottomrule
	\end{tabular}}
\end{table}

In Tabelle~\ref{tab: results} sind die Ergebnisse für $C_V$ eingetragen und in Abbildung~\ref{tab: results} graphisch dargestellt. 
Der rot eingefärbte Wert weicht deutlich von dem zu erwartenden Verlauf ab. Dies lässt sich durch Betrachtung der 
zugehörigen Temperaturwerte in Tabelle~\ref{tab: data} bzw. deren Darstellung in Abbildung~\ref{fig: temp} diskutieren. In Abbildung~\ref{fig: temp}(b)
sind die Differenzen zwischen $T\ua{prob}$ und $T\ua{Zyl}$ für die jenigen Temperaturwerte, die in die Berechnung des 
ausreißenden Datenpunktes in Abbildung~\ref{fig: C_V} eingehen, ebenfalls in rot markiert. Beide sind relativ hoch im Vergleich zu den 
sonstigen auftretenden Abweichungen, was eine mögliche Erklärung für den fehlerhaften Datenpunkt darstellt.  

Nun soll eine gute Schätzung für die Debye-Temperatur gefunden werden. Hierzu werden die gefundenen Daten mit der 
Wertetabelle~\ref{tab: wertetabelle} aus~\cite{anleitungV47} verglichen. Einem Bestwert $C_V$ wird derjenige Quotient $\Theta / T$ zugeordnet, dessen 
zugehöriger Wert in der Tabelle die geringste Abweichung hat. Um die Fehler $\Delta C_V$ sinnvoll bei diesem 
Vorgehen zu berücksichtigen werden analog die Werte $\Theta / T$ für $C_V + \Delta C_V$ und $C_V - \Delta C_V$ bestimmt. 
Als Fehler für $\theta / T$ wird dann die maximale Abweichung dieser beiden Werte vom dem für den Bestwert bestimmten 
Wert angenommen\footnote{Konkretes Beispiel $C_V = \SI{14.3(5)}{\joule\per\kelvin\per\mol}$:
Der Bestwert liegt am nächsten an dem Wert $\num{14.4840}$ aus der Tabelle, was den Wert $\Theta_D / T = \num{3.5}$ liefert. 
$(\num{14.3} + \num{0.5})\si{\joule\per\kelvin\per\mol} = \SI{14.8}{\joule\per\kelvin\per\mol}$ ergibt $\Theta_D / T = \num{3.4}$ und
$(\num{14.3} - \num{0.5})\si{\joule\per\kelvin\per\mol} = \SI{13.8}{\joule\per\kelvin\per\mol}$ ergibt $\Theta_D / T = \num{3.7}$. Der maximale 
Abstand ist also $\Theta_D / T = \num{0.2}$. Also wird angegeben: $\Theta_D / T = \num{3.5(2)}$.

}. Durch Multiplikation mit der Temperatur kann jeweils eine Debye-Temperatur berechnet werden. 
Als mit den inversen Varianzen gewichteter Mittelwert ergibt sich
\begin{equation}
        \Theta\ua{D, exp} = \input{../analysis/results/result_T_debye_exp.tex}.
\end{equation}
Für größere $T$ wird der Fehler der ermittelten Debye-Temperaturen groß (siehe Tabelle~\ref{tab: results}), da sich die Debye-Funktion für 
$T \rightarrow \Theta_D$ für beliebige $\Theta_D$ gleich verhält. Durch die gewichtete Mittelung werden diese Werte aber nur schwach 
berücksichtigt, was man alternativ durch ein einfaches Weglassen der Werte ab einer bestimmten Temperatur hätte erreichen können.
Die Debye-Funktion mit der bestimmten Debye-Temperatur ist in Abbildung~\ref{fig: C_V} eingezeichnet. Um ein Gefühl für den 
auftretenden Fehler zu geben, ist ebenfalls die $1\sigma$-Umgebung eingezeichnet (also $f(\Theta \pm \Delta\Theta_D)$).

Abschließend sollen die theoretischen Werte für die Debye-Temperatur und die Debye-Frequenz berechnet werden.
Gemäß der Gleichungen~\eqref{} und~\eqref{} ergeben sich mit $v_T = \SI{2.26}{\kilo\meter\per\second}$
und $v_L = \SI{4.7}{\kilo\meter\per\second}$ zu 
\begin{equation}
        \omega_D \approx \SI{43.5}{\per\second} \, \text{und} \quad \Theta\ua{D, theo} \approx \SI{332.5}{\kelvin}.
\end{equation}
Die Debye-Funktion mit dem Theoriewert $\Theta\ua{D, theo}$ ist ebenfalls in Abbildung~\ref{fig: C_V} eingezeichnet.

\begin{figure}
\centering
\includegraphics[scale = 1]{../analysis/results/C_V.pdf}
\caption{Temperaturverlauf der spezifischen Wärmekapazitet bei konstantem Volumen. Neben den Messdaten sind auch die 
Debye-Funktioen mit den experimentellen und theoretischen Debye-Funktionen und das Dulong-Petit-Gesetz eingezeichnet.}
\label{fig: C_V}
\end{figure}

\begin{table}
	\centering
	\caption{Wertetabelle für die Debyefunktion aus~\cite{anleitung47}.}
	\label{tab: wertetabelle}
\resizebox{\textwidth}{!}{
	\begin{tabular}{
		S[table-format=1.0]
		S[table-format=2.4]
		S[table-format=2.4]
		S[table-format=2.4]
		S[table-format=2.4]
		S[table-format=2.4]
		S[table-format=2.4]
		S[table-format=2.4]
		S[table-format=2.4]
		S[table-format=2.4]
		S[table-format=2.4]
		}
	\toprule
		{$\frac{\Theta_D}{T}$} &
		{$0$} &
		{$1$} &
		{$2$} &
		{$3$} &
		{$4$} &
		{$5$} &
		{$6$} &
		{$7$} &
		{$8$} &
		{$9$} \\
	\midrule
		 0 &  24.9430 &  24.9310 &  24.8930 &  24.8310 &  24.7450 &  24.6340 &  24.5000 &  24.3430 &  24.1630 &  23.9610 \\
		 1 &  23.7390 &  23.4970 &  23.2360 &  22.9560 &  22.6600 &  22.3480 &  22.0210 &  21.6800 &  21.3270 &  20.9630 \\
		 2 &  20.5880 &  20.2050 &  19.8140 &  19.4160 &  19.0120 &  18.6040 &  18.1920 &  17.7780 &  17.3630 &  16.9470 \\
		 3 &  16.5310 &  16.1170 &  15.7040 &  15.2940 &  14.8870 &  14.4840 &  14.0860 &  13.6930 &  13.3050 &  12.9230 \\
		 4 &  12.5480 &  12.1790 &  11.8170 &  11.4620 &  11.1150 &  10.7750 &  10.4440 &  10.1190 &  9.8030 &  9.4950 \\
		 5 &  9.1950 &  8.9030 &  8.6190 &  8.3420 &  8.0740 &  7.8140 &  7.5610 &  7.3160 &  7.0780 &  6.8480 \\
		 6 &  6.6250 &  6.4090 &  6.2000 &  5.9980 &  5.8030 &  5.6140 &  5.4310 &  5.2550 &  5.0840 &  4.9195 \\
		 7 &  4.7606 &  4.6071 &  4.4590 &  4.3160 &  4.1781 &  4.0450 &  3.9166 &  3.7927 &  3.6732 &  3.5580 \\
		 8 &  3.4468 &  3.3396 &  3.2362 &  3.1365 &  3.0403 &  2.9476 &  2.8581 &  2.7718 &  2.6886 &  2.6083 \\
		 9 &  2.5309 &  2.4562 &  2.3841 &  2.3146 &  2.2475 &  2.1828 &  2.1203 &  2.0599 &  2.0017 &  1.9455 \\
		 10 &  1.8912 &  1.8388 &  1.7882 &  1.7393 &  1.6920 &  1.6464 &  1.6022 &  1.5596 &  1.5184 &  1.4785 \\
		 11 &  1.4400 &  1.4027 &  1.3667 &  1.3318 &  1.2980 &  1.2654 &  1.2337 &  1.2031 &  1.1735 &  1.1448 \\
		 12 &  1.1170 &  1.0900 &  1.0639 &  1.0386 &  1.0141 &  0.9903 &  0.9672 &  0.9449 &  0.9232 &  0.9021 \\
		 13 &  0.8817 &  0.8618 &  0.8426 &  0.8239 &  0.8058 &  0.7881 &  0.7710 &  0.7544 &  0.7382 &  0.7225 \\
		 14 &  0.7072 &  0.6923 &  0.6779 &  0.6638 &  0.6502 &  0.6368 &  0.6239 &  0.6113 &  0.5990 &  0.5871 \\
		 15 &  0.5755 &  0.5641 &  0.5531 &  0.5424 &  0.5319 &  0.5210 &  0.5117 &  0.5020 &  0.4926 &  0.4834 \\
	\bottomrule
	\end{tabular}}
\end{table}