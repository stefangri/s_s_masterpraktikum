\section{Auswertung}



\subsection{Vorbereitende Berechnungen}
Für spätere Berechnungen ist zunächst der Zusammenhang zwischen dem Widerstand der \ce{Pt}-Elemente und der Temperatur 
zu herzustellen. Hierzu sind im Skript~\cite{anleitung47} die Daten aus Tabelle~\ref{} gegeben, die durch die quadratische 
Funktion
\begin{equation}
    T(R) = \SI{0.00134}{\kelvin \per \ohm\squared} R^2 + \SI{2.296}{\kelvin \per \ohm} R + \SI{30.13}{\kelvin}
    \label{eq: T_R}
\end{equation}
beschrieben werden. Zudem ist für die Umrechnung zwischen $C_p$ und $C_V$ mit Gleichung~\eqref{} der Temperaturverlauf 
$\alpha(T)$ des Ausdehnungskoeffizienten von Interesse. Auch hierfür sind vorgegebene Datenpunkte aus~\cite{} entnommen 
worden und in Tabelle~\ref{} eingetragen. Die Daten werden linear interpoliert und somit Werte von $\alpha$ für 
beliebige Temperaturen zwischen $\SI{80}{\kelvin}$ und $\SI{300}{\kelvin}$ bestimmt. Von einer Betrachtung der Ungenauigkeit 
dieses Modells wird wie in~\eqref{eq: T_R} abgesehen. Eine graphische Darstellung der Messpunkte und angenommenen 
Modelle befindet sich in Abbildung~\ref{fig: alpha_T_R}.
\begin{figure}
\centering
\includegraphics[scale = 1]{../analysis/results/alpha_T_R.pdf}
\caption{(a) Darstellung der Messwerte aus Tabelle~\ref{tab: } zur Beschreibung des Zusammenhangs zwischen Temperatur 
        $T$ und Widerstand $R$.
        (b) Darstellung der Messwerte aus Tabelle~\ref{tab: } zur Bestimmung des Zusammenhangs zwischen Temperatur und
         Ausdehnungskoeffizient $\alpha$.}
\label{fig: alpha_T_R}
\end{figure}


\subsection{Bestimmung der Debye-Temperatur von Kupfer}
Die gewonnen Messdaten für den Widerstand der Probe und des Zylinders, sowie der Spannung und des Stroms
als Funktion der Zeit sind in Tabelle~\ref{} aufgeführt. Für alle Daten wird eine $1$ auf der letzten
Nachkommastelle der Messwerte als Fehler angenommen. 
Mit dem Modell dem Modell~\eqref{} ergeben sich aus den Widerständen die Temperaturen von Probe und Zylinder.
Die Temperaturverläufe sind zudem in Abbildung~\ref{}(a) gezeigt. Wie im vorherigen Abschnitt beschrieben, ist es 
wichtig, dass die Temperaturen von Zylinder und Probe stets gleich sind. Da in Abbildung~\ref{} Abweichungen nicht 
deutlich zu erkennen sind, ist in Abbildung~\ref{}(b) ebenfalls die Differnez der beiden Temperaturen 
aufgetragen. Hier wird ersichtlich, dass die Temperaturabweichungen im Mittel sehr klein gehalten werden konnte, es
jedoch vereinzelt zu Temperaturdifferenzen bis zu $\sim\!\SI{4}{\kelvin}$ gekommen ist.

Aus den Daten kann nun zunächst die Wärmekapazität unter konstantem Druck $C_p$ bestimmt werden. Hierzu werden 
die Temperaturdifferenzen und Zeitdifferenzen aus den Daten in Tabelle~\ref{} gebildet und zusammen mit den 
zugehörigen Spannungen und Strömen in Formel~\eqref{} eingesetzt. Zur Berechnung der Wärmeleitfähigkeit 
bei konstantem Volumen wird Gleichung~\eqref{} verwendet. Die hierhin auftretende Temperatur wird als Mittel 
der beiden Temperaturen, aus denen die Temperaturdifferenz für Gleichung~\eqref{} bestimmt wurde, angenommen. 
Der letzte Wert übersteigt den Wert $\SI{300}{\kelvin}$ und ist damit nicht mehr mit dem Modell für $\alpha(T)$
zu beschreiben. Er wird daher nicht berücksichtigt.

In Tabelle~\ref{} sind die Ergebnisse für $C_V$ eingetragen und in Abbildung~\ref{} graphisch dargestellt. 
Der rot eingefärbte Wert weicht deutlich von dem zu erwartenden Verlauf ab. Dies lässt sich durch Betrachtung der 
zugehörigen Temperaturwerte in Tabelle~\ref{} bzw. deren Darstellung in Abbildung~\ref{} diskutieren. In Abbildung~\ref{}(b)
sind die Differenzen zwischen $T\ua{prob}$ und $T\ua{Zyl}$ für die jenigen Temperaturwerte, die in die Berechnung des 
ausreißenden Datenpunktes in Abbildung~\ref{} eingehen, ebenfalls in rot markiert. Beide sind relativ hoch im Vergleich zu den 
sonstigen auftretenden Abweichungen, was eine mögliche Erklärung für den fehlerhaften Datenpunkt darstellt.  

Nun soll eine gute Schätzung für die Debye-Temperatur gefunden werden. Hierzu werden die gefundenen Daten mit der 
Wertetabelle~\ref{} aus~\cite{} verglichen. Einem Bestwert $C_V$ wird derjenige Quotient $\Theta / T$ zugeordnet, dessen 
zugehöriger Wert in der Tabelle die geringste Abweichung hat. Um die Fehler $\Delta C_V$ sinnvoll bei diesem 
Vorgehen zu berücksichtigen werden analog die Werte $\Theta / T$ für $C_V + \Delta C_V$ und $C_V - \Delta C_V$ bestimmt. 
Als Fehler für $\theta / T$ wird dann die maximale Abweichung dieser beiden Werte vom dem für den Bestwert bestimmten 
Wert angenommen. Durch Multiplikation mit der Temperatur kann jeweils eine Debye-Temperatur berechnet werden. 
Als mit den inversen Varianzen gewichteter Mittelwert ergibt sich: 
\begin{equation}
        \Theta\ua{D, exp} = \input{../analysis/results/result_T_debye_exp.tex}.
\end{equation}

\begin{figure}
\centering
\includegraphics[scale = 1]{../analysis/results/C_V.pdf}
\end{figure}