\section{Diskussion}
Der beobachtete Verlauf der molaren Wärmekapazität $C_V$ unter Variation der Temperatur aus Abbildung~\ref{fig: C_V}
zeigt eine deutliche Abweichung vom Dulong-Petit-Gesetzt. Die Notwendigkeit einer quantenmechanischen 
Beschreibung der Gitterschwingungen in einem Festkörper für niedrigere Temperaturen konnte damit eindeutig 
demonstriert werden. Zudem ist der Verlauf einer Debye-Funktion zu erkennen, was für die 
Anwendbarkeit des Debye-Modells auf Kupfer spricht. 

Mit dem verwendeten Verfahren wurde eine Debye-Temperatur von $\input{../analysis/results/result_T_debye_exp.tex}$
bestimmt, welche in plausibler Weise von dem 
Theorie-Wert~$\SI{332.5}{\kelvin}$ abweicht. Beide Kurven sind geeignet um den Verlauf der Daten zu beschreiben.
Sowohl mit der Methode des gewichteten Mittelwertes~\eqref{eq: T_D_exp}, als auch
mit dem arithmetschen Mittel für $T\leq \SI{170}{\kelvin}$~\eqref{eq: T_D_arith} ergibt sich im Vergleich zur Theorie 
ein zu kleiner Wert. Dies ist äquivalent dazu, dass der experimentell bestimmte Verlauf der Debye-Funktion systematisch 
oberhalb dem theoretisch zu erwartenden liegt - die spezifische Wärme also zu groß gemessen wurde. Unter der Annahme einer 
hinreichend zuverlässigen Messung von Zeit, Spannung, Strom und Temperatur ist dies nach den Formeln~\eqref{eq: C_V} 
und~\eqref{eq: C_p} auf eine vergrößerte Temperaturdifferenz $\Delta T$ zurückzuführen, die nicht durch 
die zugeführte elektrische Energie bedingt ist. Hierzu können z.B. der Wärmefluss über die Aufhängung der Probe, ein 
unvollkommenes Vakuum oder der Temperaturgradient zwischen Probe und Zylinder beitragen.  

