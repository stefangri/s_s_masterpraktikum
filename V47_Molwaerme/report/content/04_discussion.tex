\section{Diskussion}
Der beobachtete Verlauf der molaren Wärmekapazität $C_V$ unter Variation der Temperatur aus Abbildung~\ref{fig: C_V}
zeigt eine deutliche Abweichung vom Dulong-Petit-Gesetzt. Die Notwendigkeit einer quantenmechanischen 
Beschreibung der Gitterschwingungen in einem Festkörper für niedrigere Temperaturen konnte damit eindeutig 
demonstriert werden. Zudem ist der Verlauf einer Debye-Funktion zu erkennen, was für die 
Anwendbarkeit des Debye-Modells auf Kupfer spricht. 

Mit dem verwendeten Verfahren wurde eine Debye-Temperatur von $\input{../analysis/results/result_T_debye_exp.tex}$
bestimmt, welche in plausibler Weise von dem 
Theorie-Wert $\SI{332.5}{\kelvin}$ abweicht. Beide Kurven sind geeignet um den Verlauf der Daten zu beschreiben.   

